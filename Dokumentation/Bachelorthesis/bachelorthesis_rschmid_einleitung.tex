\chapter*{Einleitung}
Bei der Performance-Analyse einer Software kann es notwendig sein, das  Verhalten der Garbage Collection genauer zu betrachtet. Diese Auswertung kann im laufenden Betrieb via die Software JRockit Mission Control oder aber danach auf der Basis von erstellten Log-Dateien gemacht werden. Im zweiten Fall gibt es allerdings - zumindest für die JRockit Virtual Machine (Release 28) keine Werkzeuge zur Automation dieser Aufgaben. Die vorliegende Bachelorthesis zeigt das Konzept und die Implementation einer Analysesoftware für Garbage Collection Logs der JRockit Virtual Machine (R28). Die Software soll auch als Basis für Log-Dateien anderer Garbage Collection Algorithmen dienen.

Die Bachelorthesis ist in drei Teile gegliedert: Teil eins mit Projektbeschreibung definiert die von der Schulleitung abgenommene Aufgabenstellung. Teil zwei befasst sich mit der Umsetzung und beinhaltet \titleref{analyse_aufgabenstellung}, \titleref{grundlagen_gc}, \titleref{jrockit garbage collection}, \titleref{anforderungsanalyse}, \titleref{selection_rcp_fw}\, titleref{konzept}, \titleref{implementation} und \titleref{review}. Im dritten Teil befindet sich der Anhang mit der Bedienungsanleitung und weiteren Informationen.