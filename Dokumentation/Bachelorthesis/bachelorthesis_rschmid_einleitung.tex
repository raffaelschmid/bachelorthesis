\chapter*{Einleitung}
Bei der Performanceanalyse einer Software kann es notwendig sein, das  Verhalten der Garbage Collection genauer zu betrachten. Diese Auswertung kann mit der Software JRockit Mission Control im laufenden Betrieb oder danach auf der Basis von resultierenden Logdateien gemacht werden. Im zweiten Fall gibt es allerdings - zumindest für die JRockit virtual Machine (Laufzeitumgebung) in der Version 28 keine Werkzeuge zur Automation dieser Aufgaben. Die vorliegende Bachelorthesis zeigt das Konzept und die Implementation einer Analysesoftware für Garbage Collection Logs der JRockit virtual Machine. Die Software soll so erweiterbar sein, dass sie auch für andere Garbage Collection Algorithmen erweitert werden kann.

Die Bachelorthesis ist in fünf Teile gegliedert: im Teil eins befindet sich die Projektbeschreibung mit der Aufgabenstellung und der Analyse der Aufgabenstellung, Teil zwei enthält die Grundlagen zur Performanceanalyse und Garbage Collection, Teil drei die Anforderungsanalyse, Teil vier das Konzept und Teil fünf die Umsetzung.