\subsection{Basissoftware}
\begin{longtable}{|p{4cm}|p{10.5cm}|}
  \hline
   \textbf{Abschnitt} & \textbf{Inhalt / Erläuterung} \\\hline
   Bezeichner & UC-01\\\hline
   Name & Software installieren\\\hline
   Autoren & Raffael Schmid\\\hline
   Priorität & Wichtigkeit für Systemerfolg: gross\newline Technologisches Risiko: mittel\\\hline
   Kritikalität & gross\\\hline
   Quelle & Raffael Schmid\\\hline
   Verantwortlicher & Raffael Schmid\\\hline
   Kurzbeschreibung & Der Benutzer kann das Plugin\footnote{mit Plugin ist die Garbage Collection Analysesoftware gemeint} in seiner Entwicklungsumgebung via Eingabe einer Update-Seite installieren.\\\hline
   Auslösendes Ereignis & Anwender möchte die Auswertung einer Garbage Collection Log Datei machen.\\\hline
   Akteure & Anwender\\\hline
   Vorbedingung & Richtige Entwicklungsumgebung ist bereits installiert. Anwender ist noch nicht im Besitz des Plugins.\\\hline
   Nachbedingung & Anwender hat Plugin fertig installiert.\\\hline
   Ergebnis & Entwicklungsumgebung ist bereit für Garbage Collection Auswertungen der JRockit Garbage Collection Log Dateien.\\\hline
   Hauptszenario & 
         \begin{enumerate}
		\item Anwender Startet Entwicklungsumgebung
		\item Anwender gibt Update-Seite an
		\item Anwender selektiert zu installierendes Packet
		\item Plugin wird installiert	
 	\end{enumerate}
	\\\hline
   Alternativszenarien & Anwender installiert Plugin aus lokalem Dateisystem\\\hline
   Ausnahmeszenarien & -\\\hline
   Qualitäten & -\\\hline
\caption{Use-Case: Software installieren}
\end{longtable}

\begin{longtable}{|p{4cm}|p{10.5cm}|}
\hline
   \textbf{Abschnitt} & \textbf{Inhalt / Erläuterung} \\\hline
   Bezeichner & UC-02\\\hline
   Name & Software updaten\\\hline
   Autoren & Raffael Schmid\\\hline
   Priorität & Wichtigkeit für Systemerfolg: gross\newline Technologisches Risiko: mittel\\\hline
   Kritikalität & Mittel\\\hline
   Quelle & Raffael Schmid\\\hline
   Verantwortlicher & Raffael Schmid\\\hline
   Kurzbeschreibung & Der Benutzer kann das Plugin updaten.\\\hline
   Auslösendes Ereignis & Anwender hat das Plugin bereis zu einem früheren Zeitpunkt installiert. Sofern ein neues Update dieses Plugins vorhanden ist, möchte er dieses installieren.\\\hline
   Akteure & Anwender, Update-Server\\\hline
   Vorbedingung & Richtige Entwicklungsumgebung und Garbage Collection Log Analysis Plugin ist bereits in einer früheren Version installiert.\\\hline
   Nachbedingung & Plugin ist auf dem neusten Stand.\\\hline
   Ergebnis & Entwicklungsumgebung und Plugin sind auf dem neusten Stand für Garbage Collection Auswertungen.\\\hline
   Hauptszenario & 
	\begin{enumerate}
		\item Anwender Startet Entwicklungsumgebung
		\item Anwender sucht nach neuen Updates
		\item Anwendung findet neue Updates
		\item Anwender selektiert Packete die er updaten möchte
		\item Plugin wird aktualisiert
	\end{enumerate}
	\\\hline
   Alternativszenarien & Anwender updated Plugin aus lokalem Dateisystem\\\hline
   Ausnahmeszenarien & -\\\hline
   Qualitäten & -\\\hline
\caption{Use-Case: Software updaten}
\end{longtable}

\begin{longtable}{|p{4cm}|p{10.5cm}|}
\hline
   \textbf{Abschnitt} & \textbf{Inhalt / Erläuterung} \\\hline
   Bezeichner & UC-03\\\hline
   Name & Garbage Collection Log Datei einlesen\\\hline
   Autoren & Raffael Schmid\\\hline
   Priorität & Wichtigkeit für Systemerfolg: gross\newline Technologisches Risiko: gross\\\hline
   Kritikalität & gross\\\hline
   Quelle & Raffael Schmid\\\hline
   Verantwortlicher & Raffael Schmid\\\hline
   Kurzbeschreibung & Der Benutzer kann eine sich auf seinem Rechner befindende Garbage Collection Log Datei einlesen. Die Daten befinden sich noch im unstrukturierten Zustand. Erst der Parse-Prozess für das jeweilige Log-Format bringt es in eine strukturierte Form.\\\hline
   Auslösendes Ereignis & Anwender möchte Auswertung starten.\\\hline
   Akteure & Anwender, Log Datei Importer\\\hline
   Vorbedingung & Log Datei befindet sich auf dem Rechner und ist in einem der unterstützten Formate. Garbage Collection Log Analysis Plugin ist vollständig installiert. Entwicklungsumgebung ist gestartet.\\\hline
   Nachbedingung & \\\hline
   Ergebnis & Die Log Datei befindet sich im Memory und steht für das Parsen durch die spezifische Erweiterung (JRockit) zur Verfügung.\\\hline
   Hauptszenario & 
	\begin{enumerate}
		\item Anwender öffnet Import-Dialog
		\item Anwender navigiert zur Datei
		\item Anwender startet Import-Prozess
	\end{enumerate}
	\\\hline
   Alternativszenarien & -\\\hline
   Ausnahmeszenarien & -\\\hline
   Qualitäten & Performance (QRQ-F-04)\\\hline
\caption{Use-Case: Garbage Collection Log Datei importieren}
\end{longtable}

\begin{longtable}{|p{4cm}|p{10.5cm}|}
\hline
   \textbf{Abschnitt} & \textbf{Inhalt / Erläuterung} \\\hline
   Bezeichner & UC-04\\\hline
   Name & Garbage Collection Log Datei parsen\\\hline
   Autoren & Raffael Schmid\\\hline
   Priorität & Wichtigkeit für Systemerfolg: gross\newline Technologisches Risiko: gross\\\hline
   Kritikalität & gross\\\hline
   Quelle & Raffael Schmid\\\hline
   Verantwortlicher & Raffael Schmid\\\hline
   Kurzbeschreibung & Die sich im Arbeitsspeicher befindenden unstrukturierten Daten werden durch den Parse-Vorgang in eine Dömänen-Struktur gewandelt. \\\hline
   Auslösendes Ereignis & Datei wurde fertig eingelesen.\\\hline
   Akteure & JRockit Garbage Collection Log Parser\\\hline
   Vorbedingung & Datei befindet sich im Arbeitsspeicher.\\\hline
   Nachbedingung & Kein Fehler ist aufgetreten.\\\hline
   Ergebnis & Domänenmodell ist fertig abgefüllt. \\\hline
   Hauptszenario & Jede einzelne Zeile in der Log-Datei wird analysiert und wenn nötig werden die wichtigen Daten extrahiert. Die Daten werden in das für diese Log Datei angelegte Domänenmodell geschrieben.
	\\\hline
   Alternativszenarien & -\\\hline
   Ausnahmeszenarien & -\\\hline
   Qualitäten & Performance (QRQ-F-04), Korrektheit (QRQ-F-05)\\\hline
\caption{Use-Case: Garbage Collection Log Datei importieren}
\end{longtable}

\begin{longtable}{|p{4cm}|p{10.5cm}|}
\hline
   \textbf{Abschnitt} & \textbf{Inhalt / Erläuterung} \\\hline
   Bezeichner & UC-05\\\hline
   Name & Standardauswertung anzeigen\\\hline
   Autoren & Raffael Schmid\\\hline
   Priorität & Wichtigkeit für Systemerfolg: gross\newline Technologisches Risiko: gross\\\hline
   Kritikalität & gross\\\hline
   Quelle & Raffael Schmid\\\hline
   Verantwortlicher & Raffael Schmid\\\hline
   Kurzbeschreibung & Für eine schnelle Übersicht steht eine Standard-Auswertung zur Verfügung. Dieser soll eine kurze Übersicht über die Garbage Collection geben und beinhaltet zwei Charts (Heap Benutzung, Dauer Garbage Collection). \\\hline
   Auslösendes Ereignis & Die Applikation hat die Datei fertig eingelesen und geparst.\\\hline
   Akteure & Anwender, Log Datei Analyzer, Report Engine\\\hline
   Vorbedingung & Die Log Datei wurde ohne Fehler eingelesen und befindet sich im strukturierten Format im Arbeitsspeicher.\\\hline
   Nachbedingung & -\\\hline
   Ergebnis & Dem Benutzer wird ein Analsye-Screen angezeigt.\\\hline
   Hauptszenario & 
	\begin{enumerate}
		\item Applikation hat die Log-Datei fertig importiert.
		\item Dem Benutzer wird ein Screen mit verschiedenen Tabs angezeigt. Auf jedem Tab wird dem Benutzer eine unterschiedliche Sicht auf die Daten gezeigt.
	\end{enumerate}
	\\\hline
   Alternativszenarien & Benutzerdefinierte Auswertung\\\hline
   Ausnahmeszenarien & -\\\hline
   Qualitäten & Korrektheit (QRQ-F-05)\\\hline
   Erweiterungen & UC-05.1, UC-05.2, UC-05.3, UC-05.4 \\\hline
\caption{Use-Case: Standardausertung anzeigen}
\end{longtable}

\begin{longtable}{|p{4cm}|p{10.5cm}|}
\hline
   \textbf{Abschnitt} & \textbf{Inhalt / Erläuterung} \\\hline
   Bezeichner & UC-05.1\\\hline
   Name & Anzeige Statistik Übersicht\\\hline
   Autoren & Raffael Schmid\\\hline
   Priorität & Wichtigkeit für Systemerfolg: gross\newline Technologisches Risiko: gross\\\hline
   Kritikalität & gross\\\hline
   Quelle & Raffael Schmid\\\hline
   Verantwortlicher & Raffael Schmid\\\hline
   Kurzbeschreibung & Der Analyse-Screen wurde geöffnet, dem Benutzer zeigen sich unterschiedliche Tabs. Auf dem ersten befinden sich verschiedene statistische Auswertungen der Log Datei:
   \begin{itemize}
	\item Übersicht und Grösse der verschiedenen Bereiche auf dem Heap: Initiale Grösse, endgültige Grösse
	\item Aktivitäten des Garbage Collectors: Anzahl Young Collections, Anzahl Old Collections
	\item Anzahl Garbage Collector Events (Bsp: Change Garbage Collector Strategy, etc.)
	\item Garbage Collection Zeiten (Total, Durchschnittliche, Zeit in Old Generation Garbage Collection, Prozentuale Zeit in Old Generation Garbage Collection)
   \end{itemize}
 \\\hline
   Qualitäten & Korrektheit (QRQ-F-05)\\\hline
\caption{Use-Case: Anzeige Statistik Übersicht}
\end{longtable}

\begin{longtable}{|p{4cm}|p{10.5cm}|}
\hline
   \textbf{Abschnitt} & \textbf{Inhalt / Erläuterung} \\\hline
   Bezeichner & UC-05.2\\\hline
   Name & Anzeige Heap Benutzung\\\hline
   Autoren & Raffael Schmid\\\hline
   Priorität & Wichtigkeit für Systemerfolg: gross\newline Technologisches Risiko: gross\\\hline
   Kritikalität & gross\\\hline
   Quelle & Raffael Schmid\\\hline
   Verantwortlicher & Raffael Schmid\\\hline
   Kurzbeschreibung & Die Heap Usage (Heap Benutzung) zeigt dem Benutzer anhand einer Grafik, zu welchem Zeitpunkt wieviel Speicher des Heaps verwendet wurde. Zusätzlich werden die Zeitpunkte inklusive entsprechendem Typ (Old- / Young-Collection) der Garbage Collection angezeigt.  \\\hline
   Qualitäten & Korrektheit (QRQ-F-05)\\\hline
\caption{Use-Case: Anzeige Heap Benutzung}
\end{longtable}

\begin{longtable}{|p{4cm}|p{10.5cm}|}
\hline
   \textbf{Abschnitt} & \textbf{Inhalt / Erläuterung} \\\hline
   Bezeichner & UC-05.3\\\hline
   Name & Anzeige Dauer Garbage Collection\\\hline
   Autoren & Raffael Schmid\\\hline
   Priorität & Wichtigkeit für Systemerfolg: mittel\newline Technologisches Risiko: mittel\\\hline
   Kritikalität & Mittel\\\hline
   Quelle & Raffael Schmid\\\hline
   Verantwortlicher & Raffael Schmid\\\hline
   Kurzbeschreibung & Für jede Garbage Collection ist innerhalb der Log Datei einen Eintrag mit den Informationen, wie viel Speicher von toten Objekten befreit wurde und wie lange die Collection gedauert hat. In diesem Report geht es um die Darstellung dieser Daten.\\\hline
   Qualitäten & Korrektheit (QRQ-F-05)\\\hline
\caption{Use-Case: Anzeige Dauer Garbage Collection}
\end{longtable}

\begin{longtable}{|p{4cm}|p{10.5cm}|}
\hline
   \textbf{Abschnitt} & \textbf{Inhalt / Erläuterung} \\\hline
   Bezeichner & UC-A06\\\hline
   Name & Benutzerdefiniert Auswertung (Profil)\\\hline
   Autoren & Raffael Schmid\\\hline
   Priorität & Wichtigkeit für Systemerfolg: niedrig\newline Technologisches Risiko: niedrig\\\hline
   Kritikalität & Niedrig\\\hline
   Quelle & Raffael Schmid\\\hline
   Verantwortlicher & Raffael Schmid\\\hline
   Kurzbeschreibung & Der Benutzer kann ein eigenes Analysefenster mit verschiedenen Charts anlegen. Die Charts bestehen wiederum aus verschiedenen Serien\footnote{Eine Serie definiert welche Daten auf der X- respektive Y-Achse angezeigt werden sollen.}.\\\hline
   Qualitäten & Korrektheit (QRQ-F-05)\\\hline
   Alternativszenarien & -\\\hline
   Ausnahmeszenarien & -\\\hline
   Qualitäten & Korrektheit (QRQ-F-05)\\\hline
   Erweiterungen & UC-A06.1, UC-A06.2, UC-A06.3\\\hline

\caption{Use-Case: Benutzerdefinierte Auswertung }
\end{longtable}

\begin{longtable}{|p{4cm}|p{10.5cm}|}
\hline
   \textbf{Abschnitt} & \textbf{Inhalt / Erläuterung} \\\hline
   Bezeichner & UC-06.1\\\hline
   Name & Chart definieren\\\hline
   Autoren & Raffael Schmid\\\hline
   Priorität & Wichtigkeit für Systemerfolg: niedrig\newline Technologisches Risiko: niedrig\\\hline
   Kritikalität & Niedrig\\\hline
   Quelle & Adrian Hummel (Performance Engineer)\\\hline
   Verantwortlicher & Raffael Schmid\\\hline
   Kurzbeschreibung & Der Benutzer erstellt auf dem Profil ein neues Chart und definiert was darauf angezeigt werden soll.\\\hline
   Qualitäten & -\\\hline
\caption{Use-Case: Chart definieren }
\end{longtable}

\begin{longtable}{|p{4cm}|p{10.5cm}|}
\hline
   \textbf{Abschnitt} & \textbf{Inhalt / Erläuterung} \\\hline
   Bezeichner & UC-06.2\\\hline
   Name & Auswertungsprofil speichern\\\hline
   Autoren & Raffael Schmid\\\hline
   Priorität & Wichtigkeit für Systemerfolg: niedrig\newline Technologisches Risiko: niedrig\\\hline
   Kritikalität & Niedrig\\\hline
   Quelle & Adrian Hummel (Performance Engineer)\\\hline
   Verantwortlicher & Raffael Schmid\\\hline
   Kurzbeschreibung & Profile können gespeichert werden und bleiben über einzelne Sitzungen bestehen, ausser der Benutzer löscht es explizit.\\\hline
   Qualitäten & -\\\hline
\caption{Use-Case: Profil speichern, löschen }
\end{longtable}

\begin{longtable}{|p{4cm}|p{10.5cm}|}
\hline
   \textbf{Abschnitt} & \textbf{Inhalt / Erläuterung} \\\hline
   Bezeichner & UC-06.3\\\hline
   Name & Auswertungsprofil exportieren\\\hline
   Autoren & Raffael Schmid\\\hline
   Priorität & Wichtigkeit für Systemerfolg: niedrig\newline Technologisches Risiko: niedrig\\\hline
   Kritikalität & Niedrig\\\hline
   Quelle & Adrian Hummel (Performance Engineer)\\\hline
   Verantwortlicher & Raffael Schmid\\\hline
   Kurzbeschreibung & Die vom Anwender definierten Profile können in eine Datei exportiert werden.\\\hline
   Qualitäten & -\\\hline
\caption{Use-Case: Auswertungsprofil exportieren }
\end{longtable}

\begin{longtable}{|p{4cm}|p{10.5cm}|}
\hline
   \textbf{Abschnitt} & \textbf{Inhalt / Erläuterung} \\\hline
   Bezeichner & UC-06.4\\\hline
   Name & Auswertungsprofil importieren\\\hline
   Autoren & Raffael Schmid\\\hline
   Priorität & Wichtigkeit für Systemerfolg: niedrig\newline Technologisches Risiko: niedrig\\\hline
   Kritikalität & Niedrig\\\hline
   Quelle & Adrian Hummel (Performance Engineer)\\\hline
   Verantwortlicher & Raffael Schmid\\\hline
   Kurzbeschreibung & Exportierte Profile können via ein Kanal ausgetauscht und in der anderen Entwicklungsumgebung importiert werden. \\\hline
   Qualitäten & -\\\hline
\caption{Use-Case: Auswertungsprofil importieren }
\end{longtable}

\begin{longtable}{|p{4cm}|p{10.5cm}|}
\hline
   \textbf{Abschnitt} & \textbf{Inhalt / Erläuterung} \\\hline
   Bezeichner & UC-7\\\hline
   Name & Hilfesystem\\\hline
   Autoren & Raffael Schmid\\\hline
   Priorität & Wichtigkeit für Systemerfolg: niedrig\newline Technologisches Risiko: mittel\\\hline
   Kritikalität & Mittel\\\hline
   Quelle & Raffael Schmid\\\hline
   Verantwortlicher & Raffael Schmid\\\hline
   Kurzbeschreibung & Dem Benutzer steht eine eine indexbasierte\footnote{Generelle Hilfe mit Informationen zur Garbage Collection,Vorgehensweise bei Performance Problemen, alternative Werkzeuge, etc.} und eine kontextsensitive\footnote{Hilfe zur aktuellen View oder Aktion des Benutzers} Hilfe zur Verfügung. \\\hline
   Auslösendes Ereignis & Anwender hat Plugin installiert, weiss nicht wie eine Analyse gestartet werden kann.\\\hline
   Akteure & Anwender\\\hline
   Vorbedingung & Richtige Entwicklungsumgebung und Garbage Collection Log Analysis Plugin ist bereits in einer früheren Version installiert.\\\hline
   Nachbedingung & -\\\hline
   Ergebnis & Anwender kennt Software\\\hline
   Hauptszenario &	\begin{enumerate}
		\item \textbf{Indexbasierte Hilfe: } Der Benutzer kennt sich im Thema Garbage Collection und auf der Analyse-Software noch nicht aus. Er holt sich Hilfe über die indexbasierte Hilfe. 
		\item \textbf{Kontextsensitive Hilfe: } Der Benutzer befindet sich in einem Fenster oder möchte eine Aktion ausführen (Context), das Hilfesystem zeigt ihm dazu die notwendingen Informationen.
	\end{enumerate}
	\\\hline
   Alternativszenarien & -\\\hline
   Ausnahmeszenarien & -\\\hline
   Qualitäten & QRQ-S-03 (Internationalisierung)\\\hline
\caption{Use-Case: Hilfesystem}
\end{longtable}

