\section{Use Cases}\label{use_cases}
Dieser Abschnitt zeigt die Use Cases für die Analysesoftware, daraus wird am Ende der Anforderungsanalyse die Liste der Anforderungen ermittelt. Zur Übersicht dient das Use Case Diagramm im nächsten Abschnitt. Es definiert die Systemgrenze, Anwender und die einzelnen Use Cases. Diese sind anschliessend nach der in \cite[S. 78-79]{pohl2010basiswissen} definierten Schablone beschrieben.
\subsection{Übersicht}\label{systemfunktionalitaet}
 \begin{figure}[H]
  	\centering
    	\includegraphics[width=14cm]{images/anforderungen_use-case}
        	\caption{Systemfunktionalität als Use-Case-Diagramm}
\end{figure}
\subsection{Beschreibung}
\begin{longtable}{|p{4cm}|p{10.5cm}|}
  \hline
   \textbf{Abschnitt} & \textbf{Inhalt / Erläuterung} \\\hline
   Bezeichner & UC-01\\\hline
   Name & Software installieren\\\hline
   Autoren & Raffael Schmid\\\hline
   Priorität & Wichtigkeit für Systemerfolg: gross\newline Technologisches Risiko: mittel\\\hline
   Kritikalität & gross\\\hline
   Verantwortlicher & Raffael Schmid\\\hline
   Kurzbeschreibung & Der Benutzer kann die Software in seiner Entwicklungsumgebung installieren.\\\hline
   Akteure & Anwender, Rich Client Framework\\\hline
   Auslösendes Ereignis & Anwender möchte eine Garbage Collection Logdatei analysieren.\\\hline
   Vorbedingung & Richtige Entwicklungsumgebung ist bereits ohne Analysesoftware installiert.\\\hline
   Nachbedingung & Es sind bei der Installation keine Fehler aufgetreten.\\\hline
   Ergebnis & Entwicklungsumgebung ist bereit für Garbage Collection Auswertungen.\\\hline
   Hauptszenario & 
         \begin{enumerate}
		\item Anwender startet Entwicklungsumgebung.
		\item Anwender gibt Update-Seite an.
		\item Anwender selektiert zu installierendes Softwarepaket und entscheidet sich, diese zu installieren.
		\item Softwarepaket wird installiert.
 	\end{enumerate}
	\\\hline
   Alternativszenarien & -\\\hline
   Ausnahmeszenarien & -\\\hline
   Qualitäten & Usability\\\hline
\caption{Use-Case: Software installieren}
\end{longtable}

\begin{longtable}{|p{4cm}|p{10.5cm}|}
\hline
   \textbf{Abschnitt} & \textbf{Inhalt / Erläuterung} \\\hline
   Bezeichner & UC-02\\\hline
   Name & Software updaten\\\hline
   Autoren & Raffael Schmid\\\hline
   Priorität & Wichtigkeit für Systemerfolg: gross\newline Technologisches Risiko: mittel\\\hline
   Kritikalität & Mittel\\\hline
   Verantwortlicher & Raffael Schmid\\\hline
   Kurzbeschreibung & Der Benutzer kann die Software aus der Entwicklungsumgebung aktualisieren.\\\hline
   Akteure & Anwender, Rich Client Framework\\\hline   
   Auslösendes Ereignis & Anwender hat die Software bereits zu einem früheren Zeitpunkt installiert. Es ist eine neue Version der Software auf der Update-Seite.\\\hline
   Vorbedingung & Richtige Entwicklungsumgebung und Software wurden bereits in einer früheren Version installiert.\\\hline
   Nachbedingung & Es sind keine Fehler aufgetreten.\\\hline
   Ergebnis & Entwicklungsumgebung und Analysesoftware sind auf dem neusten Stand für Auswertungen.\\\hline
   Hauptszenario & 
	\begin{enumerate}
		\item Anwender startet Entwicklungsumgebung.
		\item Anwender sucht und findet Updates für die Analysesoftware.
		\item Anwender selektiert eines oder mehrere dieser Softwarepakete und entscheidet sich, diese zu installieren.
		\item Software wird aktualisiert.
	\end{enumerate}
	\\\hline
   Alternativszenarien & -\\\hline
   Ausnahmeszenarien & -\\\hline
   Qualitäten & Usability\\\hline
\caption{Use-Case: Software updaten}
\end{longtable}

\begin{longtable}{|p{4cm}|p{10.5cm}|}
\hline
   \textbf{Abschnitt} & \textbf{Inhalt / Erläuterung} \\\hline
   Bezeichner & UC-03\\\hline
   Name & Garbage Collection Logdatei importieren\\\hline
   Autoren & Raffael Schmid\\\hline
   Priorität & Wichtigkeit für Systemerfolg: gross\newline Technologisches Risiko: gering\\\hline
   Kritikalität & gross\\\hline
   Verantwortlicher & Raffael Schmid\\\hline
   Kurzbeschreibung & Der Benutzer kann eine sich auf dem Dateisystem befindende Logdatei importieren.\\\hline
   Akteure & Anwender, Importer\\\hline
   Auslösendes Ereignis & Anwender startet Garbage Collection Log Analyse\\\hline
   Vorbedingung & Logdatei befindet sich auf dem Rechner und ist in einem der unterstützten Formate. Die Software ist vollständig installiert und gestartet.\\\hline
   Nachbedingung & Es sind keine Fehler aufgetreten. \\\hline
   Ergebnis & Die Logdatei ist in der Ansicht Logdateien ersichtlich und kann von da im Analysefenster geöffnet werden.\\\hline
   Hauptszenario & 
	\begin{enumerate}
		\item Anwender öffnet Import-Wizard
		\item Anwender navigiert zum Ordner
		\item Anwender selektiert Datei(en) und importiert diese
	\end{enumerate}
Die importierten Dateien werden gespeichert. Bei einem Neustart der Entwicklungsumgebung bleiben die zuvor importierten Dateien erhalten.
	\\\hline
   Alternativszenarien & -\\\hline
   Ausnahmeszenarien & -\\\hline
   Qualitäten & Usability\\\hline
\caption{Use-Case: Garbage Collection Logdatei importieren}
\end{longtable}

\begin{longtable}{|p{4cm}|p{10.5cm}|}
\hline
   \textbf{Abschnitt} & \textbf{Inhalt / Erläuterung} \\\hline
   Bezeichner & UC-04\\\hline
   Name & Standardauswertung anzeigen\\\hline
   Autoren & Raffael Schmid\\\hline
   Priorität & Wichtigkeit für Systemerfolg: gross\newline Technologisches Risiko: gross\\\hline
   Kritikalität & gross\\\hline
   Verantwortlicher & Raffael Schmid\\\hline
   Kurzbeschreibung & Für eine schnelle Übersicht kann der Benutzer eine Standardauswertung öffnen. Diese soll eine kurze Übersicht über die Garbage Collection geben und beinhaltet zusätzlich zwei Charts (Heap Benutzung, Dauer Garbage Collection). \\\hline
   Akteure & Anwender, Analyzer, Reporting\\\hline
   Auslösendes Ereignis & Der Benutzer hat eine Garbage Collection Logdatei importiert und möchte nun die Analyse starten.\\\hline
   Vorbedingung & Applikation hat die Logdatei fertig importiert.\\\hline
   Nachbedingung & -\\\hline
   Ergebnis & Dem Benutzer wird ein Analyse-Screen angezeigt.\\\hline
   Hauptszenario & 
	\begin{enumerate}
		\item Bevor das Analysefenster für die Logdatei geöffnet werden kann, wird die Datei eingelesen und geparst. Das heisst, dass die rohen Daten semantisch und syntaktisch analysiert und in den Arbeitsspeicher geladen werden.
		\item Dem Benutzer wird ein Screen mit verschiedenen Tabs angezeigt. Auf jedem Tab wird dem Benutzer eine unterschiedliche Sicht auf die Daten gezeigt.
	\end{enumerate}
	\\\hline
   Alternativszenarien & 
\begin{itemize}
\item Benutzerdefinierte Auswertung
\item Format der Logdatei wird nicht unterstützt, die Datei wird nicht im Analysefenster geöffnet, dem Benutzer wird eine Fehlermeldung angezeigt.
\end{itemize}\\\hline
   Ausnahmeszenarien & -\\\hline
   Qualitäten &  Korrektheit, Performance, Usability\\\hline
   Erweiterungen & UC-04.1, UC-04.2, UC-04.3, UC-04.4 \\\hline
\caption{Use-Case: Standardausertung anzeigen}
\end{longtable}

\begin{longtable}{|p{4cm}|p{10.5cm}|}
\hline
   \textbf{Abschnitt} & \textbf{Inhalt / Erläuterung} \\\hline
   Bezeichner & UC-04.1\\\hline
   Name & Anzeige Statistik Übersicht\\\hline
   Autoren & Raffael Schmid\\\hline
   Priorität & Wichtigkeit für Systemerfolg: gross\newline Technologisches Risiko: gross\\\hline
   Kritikalität & gross\\\hline
   Verantwortlicher & Raffael Schmid\\\hline
   Kurzbeschreibung & Der Analyse-Screen wurde geöffnet, dem Benutzer zeigen sich unterschiedliche Tabs. Auf dem ersten befinden sich verschiedene statistische Auswertungen der Logdatei:
   \begin{itemize}
	\item Übersicht und Grösse der verschiedenen Bereiche auf dem Heap: Initiale Grösse, endgültige Grösse
	\item Aktivitäten des Garbage Collectors: Anzahl Young Collections, Anzahl Old Collections
	\item Anzahl Garbage Collector Events (Bsp: Wechsel der Garbage Collection Strategie, etc.)
	\item Garbage Collection Zeiten (Total, Durchschnittliche, Zeit in Old Generation Garbage Collection, Prozentuale Zeit in Old Generation Garbage Collection)
	\item Durchsatz (siehe Abschnitt \ref{gc_tuning_durchsatz})
   \end{itemize}
 \\\hline
   Qualitäten &  Korrektheit, Performance, Usability\\\hline
\caption{Use-Case: Anzeige Statistik Übersicht}
\end{longtable}

\begin{longtable}{|p{4cm}|p{10.5cm}|}
\hline
   \textbf{Abschnitt} & \textbf{Inhalt / Erläuterung} \\\hline
   Bezeichner & UC-04.2\\\hline
   Name & Anzeige Heap Benutzung\\\hline
   Autoren & Raffael Schmid\\\hline
   Priorität & Wichtigkeit für Systemerfolg: gross\newline Technologisches Risiko: gross\\\hline
   Kritikalität & gross\\\hline
   Verantwortlicher & Raffael Schmid\\\hline
   Kurzbeschreibung & Die \textit{Heap Benutzung} zeigt dem Benutzer anhand einer Grafik, zu welchem Zeitpunkt wie viel Speicher des Heaps verwendet wurde. Zusätzlich werden die Zeitpunkte inklusive entsprechendem Typ (Old- / Young-Collection) der Garbage Collection angezeigt.  \\\hline
   Qualitäten & Korrektheit, Performance, Usability\\\hline
\caption{Use-Case: Anzeige Heap Benutzung}
\end{longtable}

\begin{longtable}{|p{4cm}|p{10.5cm}|}
\hline
   \textbf{Abschnitt} & \textbf{Inhalt / Erläuterung} \\\hline
   Bezeichner & UC-04.3\\\hline
   Name & Anzeige Dauer Garbage Collection\\\hline
   Autoren & Raffael Schmid\\\hline
   Priorität & Wichtigkeit für Systemerfolg: mittel\newline Technologisches Risiko: mittel\\\hline
   Kritikalität & Mittel\\\hline
   Verantwortlicher & Raffael Schmid\\\hline
   Kurzbeschreibung & Die Anzeige \textit{Dauer Garbage Collection} zeigt dem Benutzer über die Zeit, wie lange die einzelne Garbage Collection gedauert hat.\\\hline
   Qualitäten &  Korrektheit, Performance, Usability\\\hline
\caption{Use-Case: Anzeige Dauer Garbage Collection}
\end{longtable}

\begin{longtable}{|p{4cm}|p{10.5cm}|}
\hline
   \textbf{Abschnitt} & \textbf{Inhalt / Erläuterung} \\\hline
   Bezeichner & UC-05\\\hline
   Name &Profil (benutzerdefinierte Auswertung) erstellen\\\hline
   Autoren & Raffael Schmid\\\hline
   Priorität & Wichtigkeit für Systemerfolg: niedrig\newline Technologisches Risiko: niedrig\\\hline
   Kritikalität & Niedrig\\\hline
   Verantwortlicher & Raffael Schmid\\\hline
   Kurzbeschreibung & Der Benutzer kann ein eigenes Profil erstellen. Dem Profil können eigene, benutzerdefinierte Charts hinzugefügt werden. Auf jedem Chart können unterschiedliche Serien dargestellt werden. Eine Serie definiert, welche Daten auf der X- respektive Y-Achse angezeigt werden sollen. hinzugefügt werden. Die Profile sind persistent und können exportiert wie auch importiert werden.\\\hline
   Akteure & Anwender, Analyzer, Reporting\\\hline
   Auslösendes Ereignis & Die Applikation hat die Datei fertig eingelesen und geparst.\\\hline
   Vorbedingung & Die Logdatei wurde ohne Fehler eingelesen und befindet sich im strukturierten Format im Arbeitsspeicher.\\\hline
   Nachbedingung & -\\\hline
   Ergebnis & Dem Benutzer wird ein benutzerdefiniertes Analysefenster angezeigt.\\\hline
   Hauptszenario &Der Benutzer kann ein eigenes Profil erstellen. Ein Profil besteht initial aus einem Übersichtsfenster und kann um eigene, benutzerdefinierte Charts erweitert werden. Sollte die Entwicklungsumgebung mit der Analysesoftware geschlossen werden, bleiben die erstellten Profile erhalten.\\\hline
   Alternativszenarien & -\\\hline
   Ausnahmeszenarien & -\\\hline
   Qualitäten & Korrektheit \\\hline
\caption{Use-Case: Profil (benutzerdefinierte Auswertung) erstellen }
\end{longtable}




\begin{longtable}{|p{4cm}|p{10.5cm}|}
\hline
   \textbf{Abschnitt} & \textbf{Inhalt / Erläuterung} \\\hline
   Bezeichner & UC-6\\\hline
   Name & Hilfesystem\\\hline
   Autoren & Raffael Schmid\\\hline
   Priorität & Wichtigkeit für Systemerfolg: niedrig\newline Technologisches Risiko: mittel\\\hline
   Kritikalität & Mittel\\\hline
   Verantwortlicher & Raffael Schmid\\\hline
   Kurzbeschreibung & Der Benutzer kann auf eine indexbasierte\footnote{Generelle Hilfe mit Informationen zur Garbage Collection,Vorgehensweise bei Performance Problemen, alternative Werkzeuge, etc.} und eine kontextsensitive\footnote{Hilfe zur aktuellen View oder Aktion des Benutzers} Hilfe zugreifen. \\\hline
   Akteure & Anwender, Hilfesystem\\\hline
   Auslösendes Ereignis & Anwender hat Plugin installiert, weiss nicht wie eine Analyse gestartet wird.\\\hline
   Vorbedingung & Entwicklungsumgebung und Software sind installiert.\\\hline
   Nachbedingung & -\\\hline
   Ergebnis & Anwender kennt Software\\\hline
   Hauptszenario &	\begin{itemize}
		\item \textbf{Indexbasierte Hilfe: } Der Benutzer kennt sich im Thema Garbage Collection und auf der Analysesoftware noch nicht aus. Er holt sich Hilfe über die indexbasierte Hilfe. 
		\item \textbf{Kontextsensitive Hilfe: } Der Benutzer befindet sich in einem Fenster oder möchte eine Aktion ausführen (Context), das Hilfesystem zeigt ihm dazu zusätzliche Informationen.
	\end{itemize}
	\\\hline
   Alternativszenarien & -\\\hline
   Ausnahmeszenarien & -\\\hline
   Qualitäten & Internationalisierung, Usability\\\hline
\caption{Use-Case: Hilfesystem}
\end{longtable}

