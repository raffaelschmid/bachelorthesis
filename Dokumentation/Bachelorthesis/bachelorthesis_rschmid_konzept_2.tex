\chapter{Funktionalität}\label{konzept_2}
Dieser Abschnitt orientiert sich an den funktionalen Anforderungen und konzipiert diese jeweils einzeln. 

\section{Installation (FRQ-01)}\label{installation}
Zur Installation der Software benötigt man die Eclipse-Entwicklungsumgebung in der Version 3.7. Darin integriert befindet sich ein Update-Manager, der Software-Komponenten von Lokal oder dem Netzwerk installieren kann. Auch Updates werden über diesen Mechanismus installiert. Die Analysesoftware wird via eine Update-Seite bereitgestellt. Der Software-Build durch das Continuous Integration System publiziert die Artefakte (Features, Plugins) auf einen via Internet zugänglichen Server, von welchem der Update-Manager die Software herunterlädt um anschliessend zu laden. Update-Seiten im Eclipse-Umfeld bestehen aus Features (Eclipse Feature-Projekte). Features bestehen aus unterschiedlichen Plugins (Eclipse Plugin-Projekte). Die Eclipse Runtime kann solche Softwarepakete zur Laufzeit installieren und updaten.

Die Update-Seite widerspiegelt die Struktur der Software:
\begin{itemize}
\item \textbf{Basissoftware:} Umfasst alle Plugins , die für die Basissoftware notwendig sind (Siehe Abschnitt \ref{projektstruktur}).
\item \textbf{JRockit Erweiterung: }Umfasst alle Plugins zur JRockit Erweiterung und hat zugleich die Abhängigkeit auf das Basissoftware-Feature.
\end{itemize}

\section{Update (FRQ-02)}
Der Update eines Features auf der Update-Seite ist erkennbar an der neuen Versionsnummer. Beim Starten der Software wird überprüft, ob ein sich auf dem Server befindendes Feature aktualisiert wurde. Der Benutzer kann diese im laufenden Betrieb der Applikation installieren, ein Neustart ist allerdings danach notwendig.

\section{Datei importieren (FRQ-03)}
Der Import\footnote{Der Begriff Import ist irreführend, da diese Aktion einzig dazu dient, die Datei in die Ansicht Logdateien zu kriegen. Erst mit dem Öffnen der Datei werden die Daten wirklich gelesen.} einer Logdatei findet über einen Import-Wizard statt. Der Ablauf ist folgendermassen:
\begin{enumerate}
	\item Import-Wizard öffnen
	\item Auswahl des Ordners
	\item Selektion einer oder mehrerer Logdateien
	\item Bestätigung der Eingaben
	\item Anschliessend wird die Logdatei als Instanz von \textit{IFileDescriptor} (siehe Abschnitt \ref{logdatei_domainmodel}) geladen und in der Ansicht \textit{Logdateien} angezeigt, der Inhalt ist zu diesem Zeitpunkt allerdings noch nicht geladen.
\end{enumerate}

\section{Importierte Dateien speichern (FRQ-04)}
Der im Abschnitt \ref{memento} beschriebene Mechanismus wird verwendet, damit nach einem Neustart der Entwicklungsumgebung die importierten Logdateien nicht weg sind. 

\section{Datei einlesen (FRQ-05)}
Durch den Import einer Garbage Collection Logdatei erscheinen diese im Fenster \textit{Logdateien}. Das Öffnen einer dieser Dateien lädt den Inhalt in den Arbeitsspeicher. Die Daten sind noch unstrukturiert und werden in Form des im nächsten Abschnitt definierten Domänenmodells geladen.

\subsection{Domänenmodell}\label{logdatei_domainmodel}
 \begin{figure}[H]
  	\centering
    	\includegraphics[width=16cm]{images/core_domain}
        	\caption{Domänenmodell: Eingelesene Datei}
\end{figure}
\textit{IFileDescriptor} wird für die Abstraktion der Garbage Collection Logdatei verwendet. Darin enthalten sind Metadaten wie Dateiname und Pfad sowie - wenn bereits geladen - der Inhalt der Datei. Die Abstraktion des Modells hinter einer Garbage Collection heisst \textit{AbstractJvmRun} und wird erst von der jeweiligen Erweiterung realisiert.

\section{Logdatei parsen (FRQ-06)}
Die Garbage Collection Logs der JRockit virtual Machine bestehen aus Einträgen unterschiedlicher Log Module. Diese Ausgaben können selektiv per Kommandozeile aktiviert werden (siehe Abschnit \ref{logmodule}). Für die Analyse sind nicht alle Einträge relevant, es werden nur die wichtigsten angeschaut. Die einzelnen Parser für die Garbage Collection Logdatei werden auf der Basis von regulären Ausdrücken selber implementiert. Auf die Verwendung eines Parser-Generators wird aus folgenden Gründen verzichtet:
\begin{itemize}
	\item Bei der Verwendung von Parser-Generatoren muss in der Regel eine Bibliothek zusammen mit der Software ausgeliefert werden. Aufgrund der Anforderung QRQ-S-06 (siehe Abschnitt \ref{anforderungen_software}) soll die Grösse der Software möglichst klein sein. Die Implementation eines Parser auf der Basis von Regulären Ausdrücken kann ohne zusätzliche Bibliothek gemacht werden.
	\item Parser-Generatoren sind proprietär und die Einarbeitungszeit ist gross.
\end{itemize}

\subsection{Parser}
Der Parser für die Logdateien der JRockit VM ist nach dem Chain-of-Responsibility Pattern\cite{wiki:chainOfResponsibilityPattern} aufgebaut. Dieses Pattern ermöglicht die schnelle Erweiterung (Anforderung QRQ-S-01) der Software um weitere Parser. Pro Format einer Log-Zeile kann ein Parser in die Reihe geschaltet werde. Jeder Parser extrahiert die für ihn wichtigen Informationen und aktualisiert damit das Domänenmodell. Die wichtigsten Einträge der Garbage Collection Logdatei sind die des Memory Modules und werden vom \textit{MemoryModuleParser} verarbeitet. In zukünftigen Versionen ist die Verarbeitung von Log-Einträgen anderer Log-Module denkbar (Einträge des Nursery- oder Allocation-Modules, etc.). Ablauf und Aufbau des Parsers sind in den folgenden beiden Grafiken ersichtlich:

 \begin{figure}[H]
  	\centering
    	\includegraphics[width=16cm]{images/acitivity_parse_prozess}
        	\caption{Sequenzdiagramm Aufruf Parser-Hierarchie}
\end{figure}
 \begin{figure}[H]
  	\centering
    	\includegraphics[width=16cm]{images/jrockit_log_processing}
        	\caption{Klassendiagramm Parser}
\end{figure}

\subsubsection{MemoryModuleParser}
Das Parsen innerhalb des \textit{MemoryModuleParser} (Parser der die Log-Ausgaben des Memory-Log-Modules prozessiert) wird in zwei Schritte aufgeteilt:
\begin{itemize}
\item \textbf{Lexer: }Der einzelne Logeintrag wird durch den Lexer (Tokenizer) in eine Map aus Tokens umgewandelt. Schlüssel für die einzelnen Tokens ist der \textit{TokenType}. Der Lexer wird mittels Regular Expressions implementiert. Die Werte werden mittels Gruppierungs-Funktion\footnote{Bei regulären Ausdrücken lassen sich einzelne Werte aus einem String mittels Gruppierungsfunktion extrahieren.} extrahiert. 
\item \textbf{Syntactic Analyzer: }Der syntaktische Analyser verarbeitet die vom Lexer extrahierten Tokens. Er führt eine semantische Analyse durch und speichert die Werte in strukturierter Form (Domänenmodell).
\end{itemize}

\subsection{Auswertung Logdatei}
Bereits in Abschnitt \ref{logexample} ist ein Teil einer Garbage Collection Logdatei aufgelistet. Dieser Abschnitt zeigt die für die Garbage Collection Analyse wichtigsten Informationen und Auswertungen, die mit der Analysesoftware gemacht werden können, ohne dabei aber auf die Ausgaben des Debug-Log-Levels\footnote{Der Log-Level kann für jeden einzelnen Logger eingestellt werden.} einzugehen.

\subsubsection{Garbage Collection Algorithmus}
\begin{lstlisting}[numbers=none,  framexleftmargin=0mm, caption=Logdatei: Ausgabe initialer Garbage Collection Algorithmus]
[INFO ][memory ] GC mode: Garbage collection optimized for throughput, strategy: Generational Parallel Mark & Sweep.
\end{lstlisting}
Die eigentlich erste und wichtigste Entscheidung beim Garbage Collection Tuning ist die Wahl der Strategie. Sie wird im Header der Logdatei angezeigt und entspricht entweder der Standardeinstellung oder Konfiguration durch den Anwender. Ausgewertet aus dem Eintrag werden zwei Dinge: für was die Garbage Collection optimiert ist (Durchsatz oder Pausenzeiten) und welcher Algorithmus eingestellt ist.
	
\subsubsection{Initiale und maximale Heap-Kapazität, Grösse Old- und Young-Generation}
\begin{lstlisting}[numbers=none,  framexleftmargin=0mm, caption=Logdatei: Initiale und maximale Heap-Kapazität und Grösse des Old- und Young-Generation]
[INFO ][memory ] Heap size: 65536KB, maximal heap size: 1048576KB, nursery size: 32768KB.
\end{lstlisting}
Ebenfalls im Header der Logdatei befindet sich die Angabe über die initiale und maximale Heap-Kapazität und die Grösse der Young-Generation (Nursery). Die Grösse der Old-Generation (Tenured Space) errechnet sich aus der Differenz zwischen Young-Generation und maximaler Kapazität.

\subsubsection{Young-Collection Information}
\begin{lstlisting}[numbers=none,  framexleftmargin=0mm, caption=Logdatei: Information Young-Collection]
[INFO ][memory ] [YC#660] 2.172-2.172: YC 200108KB->200147KB (233624KB), 0.001 s, sum of pauses 0.536 ms, longest pause 0.536 ms.
\end{lstlisting}
Der Abschluss einer Young-Collection wird mit dem oben aufgelisteten Log-Eintrag dokumentiert. Er beschreibt Start- und Endzeitpunkt sowie die Grösse des Heaps vor und nach der Garbage Collection. Des weiteren ist die Dauer, die Summe aller einzelnen Pausen und die längste Pause ersichtlich. 

\subsubsection{Old-Collection Information}
\begin{lstlisting}[numbers=none,  framexleftmargin=0mm, caption=Logdatei: Information Old-Collection]
[INFO ][memory ] [OC#3] 2.544-2.733: OC 233624KB->187955KB (280628KB), 0.189 s, sum of pauses 187.019 ms, longest pause 187.019 ms.
\end{lstlisting}
Die Ausgabe einer Old-Collection unterscheidet sich, ausgenommen vom Typ (OC),  nicht von der einer Young-Collection.

\subsubsection{Wechsel der Garbage Collection Strategie}
\begin{lstlisting}[numbers=none,  framexleftmargin=0mm, caption=Logdatei: Wechsel Garbage Collection Strategie]
[INFO ][memory ] [OC#6] Changing GC strategy from: singleconpar to: singleconcon, reason: Return to basic strategy.
\end{lstlisting}
Aus unterschiedlichen Gründen kann es zu einem Wechsel der Garbage Collection Strategie kommen. Dieser kann, zusammen mit dem Grund, ausgewertet werden.


\subsection{Domänenmodell JRockit Garbage Collection}\label{jrockit_domain_model}
Nach der syntaktischen und sematischen Analyse wird eine Instanz des Domänenmodells erzeugt. Dieses wurde aufgrund der Randbedingungen (siehe Abschnitt \ref{randbedingungen}) analog der Struktur der Garbage Collection in der JRockit virtual Machine abstrahiert. Die rohen Daten sind also danach über diesen Objektgraphen verfügbar. \begin{landscape}
 \begin{figure}[H]
  	\centering
        	\includegraphics[width=18.2cm]{images/jrockit_extension_domain}
	\caption{Domänenmodell: Garbage Collection (JRockit Implementation)}
\end{figure}
\end{landscape}
Die Daten der Logdatei repräsentieren einen Lauf einer JVM (\textit{JRockitJVMRun}) bestehend aus einem Heap und den darin enthaltenen Bereichen Keep-Area, Nursery und Tenured Space. Jeder dieser Bereiche hat unterschiedliche Zustände (\textit{State}). Beim Übergang (\textit{Transition}) vom einen in den anderen Zustand findet eine Garbage Collection statt. Für jeden Zustand gibt es Statusinformationen welche die aktuellen Metriken (Kapazität, Auslastung, etc.) für den entsprechenden Speicherbereich beschreiben. Das starten einer Transition wird durch einen Event ausgelöst (im Diagramm nicht ersichtlich). Events werden aufgrund von heuristischen Daten der Lauzeitumgebung ausgelöst - zum Beispiel wenn die Nursery oder die Old Generation ihre Speichergrenzen erreichen. Eine Garbage Collection besteht aus unterschiedlichen Phasen (\textit{Mark}, \textit{Sweep}, \textit{Compact}).

\section{Standardauswertung anzeigen (FRQ-07)}
Von der Analysesoftware werden Standardprofile zur Verfügung gestellt, welche vom Benutzer nicht an seine eigenen Anforderungen angepasst werden können. Ein Doppelklick auf die Logdatei startet die Standardanalyse und öffnet entsprechend das Fenster mit den Tabs \textit{Übersicht}, \textit{Heap Benutzung} und \textit{Zeit Garbage Collection}. Die Aktion bedingt ein enges Zusammenspiel zwischen Basissoftware und JRockit Erweiterung: Beim Öffnen einer Analyse wird die zuständige Extension gesucht\footnote{Extensions registrieren sich via das plugins.xml an einem Extension-Point. }, welche den Inhalt\footnote{Der Inhalt der Datei wird lazy via die Methode getContent und einem ContentReader geladen.} der Logdatei verstehen kann. Die Erweiterung ist für das Parsen und Aufbereiten der Daten und die Anzeige in einem Analysefenster zuständig. Zur Illustration dient das folgende Sequenz-Diagramm:
 \begin{figure}[H]
  	\centering
    	\includegraphics[width=16cm]{images/core_sequence_analysis}
	\caption{Sequenz-Diagramm Öffnen der Analyse}
\end{figure}
Der Ablauf beim Öffnen des Analysefensters ist folgendermassen:
\begin{enumerate}
	\item Via Context wird die Erweiterung\footnote{Erweiterungen registrieren sich als Extensions an Extension-Points. Diese Konfiguration befindet sich im plugin.xml.} gesucht.
	\item Der Analyser der Erweiterung interpretiert den Inhalt und speichert die Daten im eigenen Domänenmodell (siehe Abschnitt \ref{jrockit_domain_model}).
	\item Der Editor wird geöffnet.
\end{enumerate}

\section{Anzeige Übersicht Garbage Collection (FRQ-08)}\label{standardreport}
Der initiale Tab der Analyseseite zeigt eine Zusammenfassung der geöffneten Garbage Collection Logdatei. Die Daten werden in den Tabellen Heap Kapazität, Garbage Collection Aktivität und Gesamtstatistik angezeigt:

\subsubsection{Heap Kapazität}
\begin{itemize}
	\item Initiale und maximale Heap Kapazität
	\item Grösse Young- und Old-Space
	\item Speicherbedarf maximal und durchschnittlich: Aus den Informationen des Speicherverbrauchs vor und nach jeder Garbage Collection wird der durchschnittliche und der maximale Bedarf berechnet.
	\item Kapazität maximal und durchschnittlich: Aus den Informationen der Kapazität vor und nach jeder Garbage Collection wird die durchschnittliche und maximale Kapazität des Heaps berechnet.

\end{itemize}
\subsubsection{Garbage Collection Aktivität (Young und Old Collections)}
\begin{itemize}
	\item Letzte Garbage Collection: Zu welchem Zeitpunkt hat die letzte Garbage Collection stattgefunden.
	\item Anzahl Garbage Collections: Wie viele Garbage Collections hat es insgesamt gegeben.
	\item  Anzahl Old Collections: Wie viele Old Collections hat es gegeben.
	\item Anzahl Young Collections: Wie viele Young Collections hat es gegeben.
	\item Total Zeit der Garbage Collection: Totale Zeit, in welcher sich die Virtual Machine in der Garbage Collection befunden hat.
	\item Durchsatz der Applikation (siehe Abschnitt \ref{gc_tuning_durchsatz})

	\item Durchschnittliches Intervall in Sekunden: Durchschnittliche Pausenzeit zwischen den einzelnen Garbage Collections
	\item Durchschnittliche Dauer in Sekunden
	\item Totale Zeit der Old Garbage Collection Zyklen
	\item Totale Zeit der Young Garbage Collection Zyklen
	\item Prozentuale Zeit der Old Garbage Collection Zyklen
	\item Prozentuale Zeit der Young Garbage Collection Zyklen
\end{itemize}	

\subsubsection{Gesamtstatistik}
\begin{itemize}
	\item Dauer der Messung in Sekunden
\end{itemize}

\section{Anzeige Heap Benutzung (FRQ-09)}
Die Heap-Analyse zeigt den Verlauf des benutzten \textbf{Speichers} im Heap über die Zeit auf. Die einzelnen Garbage Collection Zyklen (Young, Old) werden verschiedenfarbig dargestellt.

\subsection{Datenquellen}
Dieser Abschnitt zeigt woher die Daten für die jeweiligen Achsen kommen:
  \begin{longtable}{|p{1.5cm}|p{5.5cm}|p{4cm}|}
\hline
  \textbf{Achse} & \textbf{Beschreibung} & \textbf{Datenquelle\footnote{Zeigt den Pfad auf, wie auf die Daten zugegriffen wird. Siehe Abschnitt \ref{jrockit_domain_model}}}\\\hline
  X-Achse & Auf der X-Achse wird die Zeit für jeden Zustand (\textit{State}) angezeigt. & State.getTimestamp()\\\hline
  Y-Achse&Auf der Y-Achse wird der Verbrauch an Arbeitsspeicher angegeben. & State.getMemoryUsed()\\\hline
\end{longtable}

\section{Anzeige Dauer Garbage Collection (FRQ-10)}
Die \textbf{Dauer} der einzelnen Garbage Collection Zyklen wird über die Zeit aufgezeigt. Die einzelnen Garbage Collection Zyklen (Young, Old) werden verschiedenfarbig dargestellt.

\subsection{Datenquellen}
Dieser Abschnitt zeigt woher die Daten für die jeweiligen Achsen kommen:  \begin{longtable}{|p{1.5cm}|p{5.5cm}|p{4cm}|}
\hline
  \textbf{Achse} & \textbf{Beschreibung} & \textbf{Datenquelle}\\\hline
  X-Achse & Auf der X-Achse wird die Zeit für jeden Zustand (\textit{State}) angezeigt. & State.getTimestamp()\\\hline
  Y-Achse&Auf der Y-Achse wird der Verbrauch an Arbeitsspeicher angegeben. & State.getTransitionEnd(). getDuration()\\\hline
\end{longtable}

\section{Profil erstellen (FRQ-11)}
Die Ansicht Profile zeigt die vom Benutzer erstellten Profile\footnote{Initial befindet sich darin allerdings nur das Standard-Profil.}. Die Konfiguration eines Analysefensters (Name, Beschreibung, Charts mit Serien, etc.) wird anhand von Profilen definiert. Es gibt folgende  Arten:
\begin{itemize}
	\item \textbf{Unveränderliches Profil:} Das Standard-Profil ist aktuell das einzige unveränderliche Profil. Es wird durch die \textit{JRockit Extension} definiert und kann nicht konfiguriert werden.
	\item \textbf{Veränderliches Profil:} Ein veränderliches Profil wird zur Speicherung des vom Benutzer definierten Analysefensters verwendet. Alle Änderungen, die der Benutzer am Analysefenster macht (Chart hinzufügen, Chart konfigurieren), werden via ein Data-Binding an das Profil propagiert. Durch das Speichern des Profils hat der Benutzer die Möglichkeit, die selbe Analyse auch zu einem späteren Zeitpunkt an der gleichen oder einer anderen Logdatei durchzuführen.
\end{itemize}

\subsection{Domänenmodell zur Persistierung der Profile}
Als Struktur zur Speicherung der Profile zusammen mit den definierten Diagrammen und den sich darauf befindenden Datenserien dient folgendes Modell:
 \begin{figure}[H]
  	\centering
    	\includegraphics[width=11cm]{images/core_domain_profiles}
        	\caption{Domänenmodell: Profile}
\end{figure}

\textit{IConfiguration} dient zur Gruppierung von Profilen. Pro Extension wird eine Konfiguration mit verschiedenen Profilen abgelegt. Innerhalb eines Profils können unterschiedliche Diagramme (\textit{IChart}) angelegt werden, welche wiederum durch Achsen (\textit{IAxis}) und deren Datenquellen \textit{IValueProvider} definiert sind. \textit{IValueProvider} definieren den Weg, wie die Daten aus dem Domänenmodell ins Chart gelesen werden.

\section{Charts definieren (FRQ-12)}
Bei der Benutzung eines veränderlichen Profils hat der Benutzer die Möglichkeit, dem Analysefenster weitere Charts respektive Diagramme hinzuzufügen oder aber bereits existierende Diagramme zu manipulieren (weitere Serien hinzufügen, Serien entfernen, etc.). Die Manipulationen des Benutzers finden auf den Chart-Objekten statt und werden durch das Data-Binding auf das dahinter liegende View-Modell propagiert. Die Administrationsoberfläche für einen Chart umfasst initial zwei Abschnitte:
\begin{itemize}
	\item Serie erstellen, definieren und hinzufügen
	 \item Serie löschen
\end{itemize}

\section{Profil speichern (FRQ-13)}
Mittels des Memento-(siehe Anhang \ref{memento}) und Visitor-Patterns\cite[S. 331]{gamma1995design} werden die Profile im Eclipse-Context gespeichert. Dies hat den Vorteil, dass man sich über das Speicher-Format keine Gedanken machen muss.

\section{Profil exportieren, importieren (FRQ-14/ FRQ-15)}
Die Analysesoftware stellt zum Sichern und Verteilen von Profilen einen Import-Export-Mechanismus bereit. Beide sind in den Eclipse Standard-Wizards ersichtlich oder können via rechtem Mausklick (Ansicht Profile) gestartet werden. Der Profile-Export und -Import basiert wie das Speichern auf dem im Abschnitt \ref{memento} beschriebenen Memento-Mechanismus. Zur Serialisierung in eine Datei wird das XMLMemento verwendet. 

\chapter{Generelle Aspekte}
Dieser Abschnitt konzipiert die nichtfunktionalen Anforderungen (Qualitätsanforderungen).

\section{Hilfesystem (FRQ-16)}\label{hilfesystem}
Das Hilfesystem der Eclipse Entwicklungsumgebung ist als Client-Server-Lösung implementiert. Beim Start der Entwicklungsumgebung wird zusätzlich ein Jetty-Server gestartet, der die Hilfedienste wie Suche und Indexierung bereitstellt.  Hilfeseiten werden als HTML-Seiten umgesetzt und vom internen Browser angezeigt. Es gibt zwei verschiedene Hilfetypen:


\begin{itemize}
\item \textbf{Indexbasierte Hilfen:} Für die generellen Informationen und Hilfen werden verschiedene Hilfeseiten basierend auf einem Index bereitgestellt. Die Inhalte sind nicht an ein Fenster oder eine Aktion des Benutzers gebunden. 
\item \textbf{Kontextsensitive Hilfen:} Hilfeseiten, welche den Bezug auf eine Aktion oder ein Fenster haben, werden über einen Schlüssel mit der jeweiligen Komponente verlinkt.
\end{itemize}

%--------------------------------------------
\section{Testabdeckung (QRQ-S-02)}\label{testing}
Bei Plugin-Projekten wird der Test-Code normalerweise in eigenen Fragment-Projekten abgelegt. Diese Test-Projekte werden allerdings nicht mit dem Feature ausgeliefert, sondern nur während dem Testen verwendet. Der Zugriff auf den Code der Implementation wird gewährleistet, indem aus dem Test-Projekt ein Fragment\footnote{Fragmente sind Erweiterungen für ein Projekt (Host) und haben vollen Zugriff auf die Klassen ihres Host-Bundles.} gemacht wird. Als Testing-Framework wird der defakto Standard JUnit verwendet. JUnit ist sowohl von allen Entwicklungsumgebungen als auch von Maven und Tycho unterstützt.

\section{Internationalisierung (QRQ-S-03))}
Die Analysesoftware soll in den Sprachen Deutsch und Englisch verfügbar sein. Die gewählte Sprache innerhalb der Erweiterung wird von der Entwicklungsumgebung übernommen\footnote{Die Entwicklungsumgebung übernimmt die Sprache der Java Lauzeitumgebung: Voreinstellung oder Auswahl über Kommandozeile (-nl de). } und kann nicht via ein Menu geändert werden. 
Der Eclipse Lokalisierungsmechanismus basiert auf der Java \textit{Locale} und löst die Ressourcen für die ausgewählte Sprache aus Properties-Dateien auf. Je nachdem wo der Text platziert ist, wird die Sprachressource auf unterschiedliche Weise aufgelöst:
\begin{itemize}
	\item \textbf{Texte, Labels im Code:} Eclipse stellt zur Externalisierung von Strings einen Wizard zur Verfügung. Die Texte werden in Properties-Dateien extrahiert und beim Starten der Applikation geladen. Die extrahierten Sprachressourcen werden im Klassenpfad abgelegt.

	\item \textbf{Texte, Labels in Deskriptoren:} Die sich in den Eclipse-Deskriptoren (plugin.xml, Manifest.MF) befindenden sprachabhängigen Plugin-Informationen wie Organisation, Name und Beschreibung werden ebenfalls in Properties-Dateien abgelegt. Diese werden bei der Installation oder während einem Update angezeigt. Der Ort und Name dieser Dateien ist per Konvention \textit{\textbackslash \$\{plugin\}\textbackslash OSGI-INF\textbackslash I10n\textbackslash bundle\_\$\{lang\}.properties}. Der Inhalt wird vom Eclipse-Framework geladen.

	\item \textbf{Hilfesystem:} Die HTML-Hilfeseiten können ebenfalls in unterschiedlichen Sprachen definiert und angezeigt werden. Siehe Abschnit \ref{hilfesystem}.
\end{itemize} 

\section{Usability (QRQ-S-04))}
Einige der Funktionalitäten der Analysesoftware wie beispielsweise das Einlesen und Parsen der Logdateien dauern lange. Der Benutzer benötigt eine Statusinformation. Operationen dieser Art werden mittels des Eclipse \textit{IProgressService} gestartet. Dies hat für die Applikation und den Benutzer folgende Vorteile:
\begin{itemize}
	\item \textbf{Nebenläufigkeit:} Die Applikation startet die Arbeit in einem eigenen nicht-UI Thread\footnote{Einem Thread, der nicht für das Zeichnen des Benutzerinterfaces verwendet wird.}, sodass es nicht zu Nebeneffekten wie einem eingefrorenen Bildschirm kommt. Der Benutzer kann die Fortschrittsanzeige minimieren und mit der Applikation weiterarbeiten.
	\item \textbf{Fortschrittsanzeige: } Die Applikation teilt dem Benutzer über eine Anzeige mit, bei welcher Position sich der Prozess befindet und wie viel Arbeit prozentual bereits gemacht wurde.
	\item \textbf{Unterbrechbarkeit: } Der Prozess erkundigt sich periodisch bei der Monitoring-Komponente ob er durch den Benutzer abgebrochen wurde. Sobald dies der Fall wäre, würde er die Arbeit beenden und das bereits Erledigte aufräumen.
\end{itemize}


\section{Korrektheit (QRQ-S-05))}
Um mit Java-Applikationen genaue Werte zu berechnen wird unter \cite{bloch2008effective} empfohlen, die Klasse \textit{BigDecimal} anstelle von \textit{Double} und \textit{Long} zu verwenden. Die in der Analysesoftware angezeigten Werte haben eine Genauigkeit von mindestens einem Zehntel (0.1).
