\chapter{Auswahl Frameworks und Komponenten}\label{selection_rcp_fw}
\section{Rich Client Frameworks}
Grundsätzlich käme für die Entwicklung der Analyse-Software auch die Java GUI-Bibliothek Swing in Frage. Viele Anforderungen die im Zusammenhang mit Desktop-Applikationen entstehen (Deployment, Installation, Update, Modularisierung, Internationalisierung, etc.), müssen dann allerdings von neuem entwickelt werden. Es macht deshalb Sinn bei der Evaluation der Bibliothek nur Rich Client Frameworks zu berücksichtigen. Von denen kommen aufgrund der funktionalen Anforderung FRQ-X01 (Offline Betriebsmodus) und FRQ-X02 (Installation als Erweiterung) Eclipse RCP und Netbeans RCP in Frage. Während Eclipse RCP insbesondere als Entwicklungsumgebung ein weit verbreitetes Framework ist und die Entwicklung aktuell in zwei verschiedenen Versionen (3.x / 4.x) vorangetrieben wird, findet man Netbeans und deren Rich Client Plattform eher seltener. In der folge werden die einzelnen Frameworks kurz beschrieben und dann anhand der Anforderungen verglichen.


\subsection{Übersicht}
\subsubsection{Eclipse RCP}
Bis zur Version 2.1 war Eclipse bekannt als eine Open Source Entwicklungsumgebung für Programmierer. Der Vorgänger hiess Visual Age vor Java und wurde von IBM entwickelt. 

Auf die Version 3.0 wurde die Architektur von Eclipse relativ stark umgestellt und modularisiert. Nun handelt es sich um einen relativ kleinen Kern der Applikation, der die eigentliche Funktionalität der Applikation als Plugins lädt. Diese Funktionalität basiert auf Eclipse Equinox, einer Implementation der OSGi Spezifikation. Die grafischen Benutzeroberflächen sind in SWT implementiert. Eclipse ist für 14 verschiedene Systeme und Architekturen bereitgestellt und gilt somit als plattformunabhängig \cite{wiki:eclipse}. 

Die Plattform kann nun auch als Framework zur Entwicklung von Desktop Applikationen oder Plugins für die Entwicklungsumgebung entwickeln verwendet werden. Bei eigenen Desktop-Anwendungen spricht man auch von Eclipse-RCP Anwendungen.



\subsubsection{Netbeans RCP}
Wie bei Eclipse RCP handelt es sich bei Netbeans RCP ebenfalls um ein Framework zur Entwicklung von Desktop Anwendungen. Dem Entwickler wird ein API für typische Anforderungen in diesem Bereich zur Verfügung gestellt. Der Kern der Netbeans Plattform besteht ebenfalls aus einem Modul-Loader und im Bereich der grafischen Benutzeroberfläche wird Swing verwendet.

\subsection{Auswertung Rich Client Frameworks}
Dieser Abschnitt illustriet die Bewertung und Gewichtung und wertet anschliessend die drei zur Auswahl stehenden Rich Client Plattformen tabellarisch aus. 
\subsubsection{Bewertung}
Für die Bewertung werden die Punktezahlen  folgendermassen vergeben:
\begin{longtable}{|l|c|c|c|c|c|}\hline
 \textbf{Bewertung} & sehr schlecht & schlecht & mittel & gross & sehr gross\\\hline
 \textbf{Punktzahl} & 1 & 2 & 3 & 4 & 5\\\hline
 \caption{Schema Vergabe der Punkte}
\end{longtable}

\subsubsection{Gewichtung}
Der Wert der Gewichtung wird folgendermassen in eine Zahl umgewandelt: 
\begin{longtable}{|l|c|c|c|c|c|}\hline
 \textbf{Priorität} & sehr klein & klein & mittel & gross & sehr gross\\\hline
 \textbf{Gewicht} & 1 & 2 & 3 & 4 & 5\\\hline
 \caption{Schema Gewichtung der Prioritäten}
\end{longtable}


Die erste und zweite Spalte zeigen die Anforderungen aus Abschnitt \titleref{anforderungen_framework} mit der jeweiligen Gewichtung. Die weiteren Spalten zeigen die erreichte Punktzahl zusammen mit dem aus Punktzahl und Gewicht errechnete Produkt.
\begin{longtable}{|p{3cm}|c|c|c|c|c|}\hline
 \textbf{Anforderung} & \textbf{Nummer} &  \textbf{Gewicht.} & \textbf{Eclipse 3.x} & \textbf{Eclipse 4.x} &  \textbf{Netbeans 3.x}\\\hline
   \textbf{Verbreitung} & (QRQ-F-01) & 4 & 5 (20) & 1 (4) & 2 (8)\\\hline
   \textbf{Unterstützung Plattformunab-hängigkeit} & (QRQ-F-02) & 4 & 5 (20) & 5 (20) & 5 (20)\\\hline
   \textbf{Unterstützung Lokalisation Support} & (QRQ-F-03) & 2 & 5 (10) & 5 (10) & 5 (10) \\\hline
   \textbf{Unterstützung Modularisierung} & (QRQ-F-04) & 3 & 5 (15) & 5 (15) & 5 (15) \\\hline
   \textbf{Total} & - & - & \textbf{65} & 49 & 53\\\hline
    \caption{Auswertung Rich Client Frameworks}
\end{longtable}


\subsection{Entscheid}\label{rcp_entscheid}
Ausschlaggebend für die Wahl der Rich Client Plattform ist die Verbreitung. Trotzdem dass die Technologie und die Architektur der drei Frameworks teilweise grosse Unterschiede aufweisen, unterscheiden sie sich im Funktionsumfang nur unwesentlich. 

