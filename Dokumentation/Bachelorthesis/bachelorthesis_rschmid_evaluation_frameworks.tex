\chapter{Auswahl Frameworks und Komponenten}\label{selection_rcp_fw}
\section{Einleitung}
Die Evaluation der Komponenten wird auf der Basis der Anforderungen gemacht. Die jeweiligen Punktezahlen werden anhand folgender Schemas ausgerechnet.
\subsubsection{Bewertung}
Die Bewertung wird aus dem Erfahrungswert und folgender Tabelle in eine Punktzahl umgewandelt:
\begin{longtable}{|l|c|c|c|c|c|}\hline
 \textbf{Bewertung} & sehr schlecht & schlecht & mittel & gross & sehr gross\\\hline
 \textbf{Punktzahl} & 1 & 2 & 3 & 4 & 5\\\hline
 \caption{Schema Vergabe der Punkte}
\end{longtable}

\subsubsection{Gewichtung}
Die Priorität berechnet sich aus der Priorität der Anforderung und folgender Tabelle:\begin{longtable}{|l|c|c|c|c|c|}\hline
 \textbf{Priorität} & sehr klein & klein & mittel & gross & sehr gross\\\hline
 \textbf{Gewicht} & 1 & 2 & 3 & 4 & 5\\\hline
 \caption{Schema Umwandlung Priorität in Gewicht}
\end{longtable}
\subsubsection{Berechnung der Punkte}
Die Punktezahl ergibt sich jeweils aus dem Produkt von Bewertung und Gewicht:
\begin{center}
\begin{tabular}{|c|}\hline
Punktezahl = Bewertung x Gewicht \\\hline
\end{tabular}
\end{center}
\section{Rich Client Framework}
Grundsätzlich käme für die Entwicklung der Analyse-Software auch die Java GUI-Bibliothek Swing in Frage. Viele Anforderungen die im Zusammenhang mit Desktop-Applikationen entstehen (Deployment, Installation, Update, Modularisierung, Internationalisierung, etc.), müssen dann allerdings neu entwickelt werden. Bei der Evaluation der Bibliothek werden darum nur Rich Client Frameworks berücksichtigt. Von denen kommen aufgrund der funktionalen Anforderung FRQ-F-05 (Offline Betriebsmodus) und FRQ-F-06 (Installation als Erweiterung) Eclipse RCP und Netbeans RCP in Frage. Während Eclipse RCP insbesondere als Entwicklungsumgebung ein weit verbreitetes Framework ist und die Entwicklung aktuell in zwei verschiedenen Versionen (3.x / 4.x) vorangetrieben wird, findet man Netbeans und deren Rich Client Plattform eher selten. In der Folge werden die drei Frameworks kurz beschrieben und dann anhand der Anforderungen verglichen.


\subsection{Übersicht}
\subsubsection{Eclipse RCP}
Bis zur Version 2.1 war Eclipse bekannt als eine Open Source Entwicklungsumgebung für Programmierer. Der Vorgänger hiess Visual Age vor Java und wurde von IBM entwickelt. 

Auf die Version 3.0 wurde die Architektur von Eclipse relativ stark umgestellt und modularisiert. Nun handelt es sich um einen relativ kleinen Kern, der die eigentlichen Funktionalitäten der Applikation als Plugins lädt. Der beschriebene Mechanismus basiert auf Eclipse Equinox, einer Implementation der OSGi Spezifikation. Die grafischen Benutzeroberflächen werden in SWT implementiert. Eclipse ist für 14 verschiedene Systeme und Architekturen bereitgestellt und gilt somit als plattformunabhängig\cite{wiki:eclipse}. 

Die Plattform kann nun auch als Framework zur Entwicklung von Desktop Applikationen oder Plugins für die Entwicklungsumgebung  verwendet werden.


\subsubsection{Netbeans RCP}
Bei Netbeans RCP handelt es sich ebenfalls um ein Framework zur Entwicklung von Desktop Anwendungen. Der Kern der Netbeans Plattform besteht ebenfalls aus einem Modul-Loader und im Bereich der grafischen Benutzeroberfläche wird Swing verwendet. Man findet Netbeans als Rich Client Framework eher selten.

\subsection{Auswertung Rich Client Frameworks}
Die erste und zweite Spalte zeigen die Anforderungen aus Abschnitt \ref{anforderungen_framework}. Die dritte Spalte enthält das aus der Priorität abgeleitete Gewicht (siehe Abschnitt Gewichtung). Die weiteren Spalten zeigen pro Produkt die Bewertung zusammen mit dem Total der Punkte.
\begin{longtable}{|p{3cm}|c|c|c|c|c|}\hline
 \textbf{Anforderung} & \textbf{Nummer} &  \textbf{Gewicht.} & \textbf{Eclipse 3.x} & \textbf{Eclipse 4.x} &  \textbf{Netbeans 3.x}\\\hline
   Verbreitung & (QRQ-F-01) & 4 & 5 (20) & 1 (4) & 2 (8)\\\hline
   Unterstützung Plattformunab-hängigkeit & (QRQ-F-02) & 4 & 5 (20) & 5 (20) & 5 (20)\\\hline
   Unterstützung Lokalisation Support & (QRQ-F-03) & 2 & 5 (10) & 5 (10) & 5 (10) \\\hline
   Unterstützung Modularisierung & (QRQ-F-04) & 3 & 5 (15) & 5 (15) & 5 (15) \\\hline
   \textbf{Total} & & & \textbf{65} & \textbf{49} & \textbf{53}\\\hline
    \caption{Auswertung Rich Client Frameworks}
\end{longtable}

\subsection{Entscheid}\label{rcp_entscheid}
\textbf{Ausschlaggebend für die Wahl der Rich Client Plattform ist die Verbreitung.} Trotz der Unterschiede in Architektur und den verwendeten Technologien unterscheiden sich die Frameworks im Funktionsumfang nur unwesentlich.


\section{Charting Bibliothek}
Für die grafische Darstellung der Daten (Charts) wird eine Bibliothek verwendet. Sehr oft wird dafür Eclipse BIRT oder JFreeChart eingesetzt.
\subsection{Übersicht}
\subsubsection{BIRT}
BIRT steht für Business Intelligence and Reporting Tools und stellt Business-Intelligence- und Reporting-Funktionalität für Rich Clients zur Verfügung. BIRT ist Lizenziert unter der Eclipse Public License und ist daher open-source. BIRT ist wie Eclipse ein Top-Level-Softwareprojekt der Eclipse Foundation. Die Software besteht aus zwei Teilen:
\begin{itemize}
\item \textbf{Report Designer: }Mit dem Report Designer können Benutzer oder Administratoren ihre Reports erstellen und anpassen. 
\item \textbf{Charting Engine: }Die Charting Engine generiert aus den Daten und den definierten Reports, Diagrammen die Charts.
\end{itemize} 

Der Umfang der Software ist mit über 100 Megabyte für kleinere Projekte eher ungeeignet.

\subsubsection{JFreeChart}
JFreeChart ist ein Open-Source-Framework mit welchem eine Vielzahl verschiedener Diagramme erzeugt werden können. JFreeChart unterstützt die Ausgabe der Diagramme in Grafiken, Swing- und RCP-Applikationen und ist mit einer Grösse von rund einem Megabyte auch für kleinere Softwareprojekte geeignet.

\subsection{Auswertung Charting-Bibliothek}
\begin{longtable}{|p{4cm}|c|c|c|c|}\hline
 \textbf{Anforderung} & \textbf{Nummer} &  \textbf{Gewicht.} & \textbf{BIRT} & \textbf{JFreeChart}\\\hline
   Grösse Softwarepacket & (QRQ-S-06) & 4 & 1 (4) & 5 (20) \\\hline
   Verbreitung & (QRQ-F-01) & 4 & 4 (16) & 5 (20) \\\hline
   \textbf{Total} & && \textbf{20}  & \textbf{40} \\\hline
    \caption{Auswertung Charting-Bibliothek}
\end{longtable}

\subsection{Entscheid}
JFreeChart eignet sich insbesondere für kleinere Client-Applikationen sehr gut zur Erstellung von Diagrammen. Die Flexibilität welche bei Birt durch den Report-Designer bereitgestellt wird, wird in der Analysesoftware nicht benötigt. Aus diesen und oben aufgelisteten Gründen wird JFreeChart als Charting-Bibliothek verwendet.
