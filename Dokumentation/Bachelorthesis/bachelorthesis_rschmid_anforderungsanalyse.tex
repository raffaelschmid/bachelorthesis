\chapter{Anforderungsanalyse}\label{anforderungsanalyse}
Die Evaluation von Rich Client Framework und Charting-Bibliothek sowie die Konzeption der funktionalen Aspekte ist getrieben durch die Anforderungsanalyse. Es standen drei Methoden zur Auswahl: Use Cases, Requirements Engineering nach IEEE 830 und User Stories. Auf Ebene der Customer Requirements\footnote{Nach Standard IEEE 830\cite{ wiki:ieee830} werden die Anforderungen in Customer Requirements (Anforderungen aus Sicht des Kunden) und Development Requirements (Anforderungen aus Sicht des Entwicklers) unterteilt. } wird die Methode der Use Cases eingesetzt. Sie zeichnet sich dadurch aus, dass man die Anforderungen nicht nur textuell sondern auch modellbasiert mit UML definieren kann. Für die Dokumentation der Development-Requirements, beinhaltet auch die Qualitätsanforderungen, wird der Standard IEEE 830 verwendet. Die damit definierten Anforderungen lassen sich gut in einzelne Entwicklungspakete verpacken. Auch die Struktur der Anforderungsanalyse wurde anhand dieser Definition aufgebaut.


\section{Einleitung}
\subsection{Zweck}
Die Anforderungsanalyse dient als Basis für die folgenden Abschnitte:

\begin{itemize}
	\item \textbf{Architektur und Konzept:}  Die dokumentierten Anforderungen dienen als Grundlage für die Architektur des Systems und das Konzept. 
	\item \textbf{Implementation:} Die Implementation der Anwendung richtet sich nach den ermittelten Anforderungen. 
	\item \textbf{Verifikation:} Der implementierte Software-Prototyp wird anhand der in diesem Abschnitt ermittelten Anforderungen bewertet. 
\end{itemize}

\subsection{Systemumfang}
Der folgende Abschnitt beschreibt die wesentlichen Teile innerhalb des Systems und des Systemkontexts. 
 \begin{figure}[H]
  	\centering
    	\includegraphics{images/systemumfang}
        	\caption{System und Systemkontext}
\end{figure}
\subsubsection{System}
Das System besteht aus den Teilen Importer, Analyzer, Reporting und Hilfesystem\footnote{Dem Benutzer wird sowohl eine generelle wie auch eine kontextsensitive Hilfe angeboten.}. Diese Teile sind Bestandteil der Entwicklung und können den Anforderungen entsprechend konzipiert und umgesetzt werden.

\subsubsection{Systemkontext}
Zum Systemkontext gehören folgende nicht veränderbare Komponenten:
\begin{itemize}
	\item \textbf{Build und Deployment, Continuous Integration:} Der Build der Software für neue Updates oder Releases wird zentral auf einem Server durchgeführt. Der Source-Code wird aus der Versionskontrolle ausgecheckt, die binären Packete werden gebildet, es wird ein für den jeweiligen Update-Mechanismus notwendiges Packet erstellt und auf den Update-Server gestellt.

	\item \textbf{Rich Client Framework:} Als Basis der Analysesoftware wird ein Rich Client Framework verwendet. Die Struktur und Architektur dieser Bibliothek beeinflusst den Aufbau der Software massgeblich.
	\item \textbf{JRockit virtual Machine:} Die Schnittstelle zur JRockit Virtual Machine findet über deren Logdateien statt. Die genaurere Beschreibung befindet sich im Abschnitt \ref{jrockitgclog}.
\end{itemize}

\subsubsection{Irrelevante Umgebung}
\begin{itemize}
	\item \textbf{Issuetracker:} Sobald die Software stabil läuft und an Tester herausgegeben wird, wird für die Verwaltung der Fehler (Bugs) und Features ein Issuetracker verwendet.
	\item \textbf{Versionskontrolle:} Der Source-Code der Applikation wird in einer Source-Code-Verwaltung abgelegt. Diese dient als Backup und zur Versionierung, Historisierung der einzelnen Artefakte. Issuetracker und Versionskontrolle arbeiten eng zusammen, so dass Issues mit der eingecheckten Version in Verbindung gebracht werden können.
\end{itemize}


\subsection{Stakeholder}
Für die Anforderungsanalyse sind im Diagramm unten nur die mit Asteriks gekennzeichneten Rollen\footnote{Verschiedene Rollen können von einer Person wahrgenommen werden.} relevant. Die anderen ab dem Zeitpunkt des Releases der Software.


\begin{figure}[H]
  	\centering
    	\includegraphics[width=13cm]{images/stakeholder_analyse}
        	\caption{Übersicht der Stakeholder}
\end{figure}

\subsection{Glossar}
Das Glossar befindet sich im Anhang \ref{glossar}. 
\subsection{Referenzen}
Die Referenzen innerhalb des Abschnitts \ref{anforderungsanalyse} befinden sich im Literaturverzeichnis.
\subsection{Übersicht}
Die Anforderungsanalyse ist folgendermassen aufgebaut: Beginnen tut sie mit der Allgemeinen Übersicht. Hier sind Architektur, Nutzer- und Zielgruppen und Randbedingungen dokumentiert. Die Customer-Requirements werden mit Use Cases definiert, die Development Requirements richten sich nach dem Standard IEEE 830.

\section{Allgemeine Übersicht}\label{allgemeine_uebersicht}
\subsection{Architekturbeschreibung}
Die Software wird als Plugin programmiert. Der Entwickler kann sich die Software als Erweiterung in seiner Entwicklungsumgebung installieren. Sobald die Software installiert ist, wird für die Arbeit keine Verbindung ins Internet mehr benötigt (offline). Die Applikation ist auf allen gängigen Betriebssystemen (Linux, Mac OSX, Windows) lauffähig.

\subsection{Nutzer und Zielgruppen}
\subsubsection{Performance Engineers}
Software Entwickler welche die Erfahrung, das Wissen und die Fähigkeit haben, um die Ursachen für Performance-Probleme zu finden. Dabei handelt es sich nicht nur um Wissen im Bereich der Softwareentwicklung, sondern auch im Bereich des Servers und des Betriebssystems. Sie haben sich auch eine strukturierte Vorgehensweise angeeignet und verfügen über die Kenntnisse der wichtigsten Werkzeuge. Durch ihr breites Wissen sind sie mit der Unterstützung von Charts, Statistiken und Reports in der Lage, Ursachen von Performanceproblemen zu finden.

\subsubsection{Java Entwickler}
Im Gegensatz zu Performance Engineers beschäftigen sich Java Entwickler vorallem mit der Entwicklung von Anwendungen und verfügen nicht direkt über Knowhow im Bereich der Performanceanalyse. Durch ihre tägliche Arbeit kommen sie mit den Anforderungen an die Performance in Kontakt. Als gut ausgebildete Ingenieure sind sie aber mit Hilfe von Werkzeugen und Dokumentation in der Lage, Performance Problemen innert nützlicher First auf den Grund zu gehen.

\subsection{Randbedingungen}\label{randbedingungen}
\subsubsection{JRockit R28}
Der Fokus der Analysesoftware stützt sich auf die Logdateien der JRockit Virtual Machine Release 28. Auswertungen der Garbage Collection von VMs anderer Hersteller werden eventuell zu einem späteren Zeitpunkt implementiert. 

\section{Use Cases}\label{use_cases}
Dieser Abschnitt zeigt die Use Cases für die Analysesoftware. Daraus werden dann im nächsten Abschnitt die Anforderungen respektive die finale Anforderungsliste abgeleitet. Als Einstieg dient das UML Use Case Diagramm, in welchem die Systemgrenzen, der Anwender (Performance Analyst) und die verschiedenen Use Cases dargestellt sind. Die einzelnen Use Cases sind dann nach der in \cite[S. 78-79]{pohl2010basiswissen} beschriebenen Schablone definiert.
\subsection{Übersicht}\label{systemfunktionalitaet}
 \begin{figure}[H]
  	\centering
    	\includegraphics[width=14cm]{images/anforderungen_use-case}
        	\caption{Systemfunktionalität als Use-Case-Diagramm}
\end{figure}
\subsection{Beschreibung}
Die definierten Use Cases leiten sich aus einer Anforderung ab, welche über die Nummer referenziert ist.
\begin{longtable}{|p{4cm}|p{10.5cm}|}
  \hline
   \textbf{Abschnitt} & \textbf{Inhalt / Erläuterung} \\\hline
   Bezeichner & UC-01\\\hline
   Name & Software installieren\\\hline
   Autoren & Raffael Schmid\\\hline
   Priorität & Wichtigkeit für Systemerfolg: gross\newline Technologisches Risiko: mittel\\\hline
   Kritikalität & gross\\\hline
   Verantwortlicher & Raffael Schmid\\\hline
   Kurzbeschreibung & Der Benutzer kann die Software in seiner Entwicklungsumgebung installieren.\\\hline
   Akteure & Anwender, Entwicklungsumgebung\\\hline
   Auslösendes Ereignis & Anwender möchte eine Garbage Collection Logdatei analysieren.\\\hline
   Vorbedingung & Richtige Entwicklungsumgebung ist bereits ohne Analysesoftware installiert.\\\hline
   Nachbedingung & Es sind keine Fehler aufgetreten.\\\hline
   Ergebnis & Entwicklungsumgebung ist bereit für Garbage Collection Auswertungen.\\\hline
   Hauptszenario & 
         \begin{enumerate}
		\item Anwender Startet Entwicklungsumgebung
		\item Anwender gibt Update-Seite an
		\item Anwender selektiert zu installierendes Softwarepaket
		\item Softwarepaket wird installiert	
 	\end{enumerate}
	\\\hline
   Alternativszenarien & -\\\hline
   Ausnahmeszenarien & -\\\hline
   Qualitäten & Usability\\\hline
\caption{Use-Case: Software installieren}
\end{longtable}

\begin{longtable}{|p{4cm}|p{10.5cm}|}
\hline
   \textbf{Abschnitt} & \textbf{Inhalt / Erläuterung} \\\hline
   Bezeichner & UC-02\\\hline
   Name & Software updaten\\\hline
   Autoren & Raffael Schmid\\\hline
   Priorität & Wichtigkeit für Systemerfolg: gross\newline Technologisches Risiko: mittel\\\hline
   Kritikalität & Mittel\\\hline
   Verantwortlicher & Raffael Schmid\\\hline
   Kurzbeschreibung & Der Benutzer kann die Software updaten.\\\hline
   Akteure & Anwender, Update-Server\\\hline   
   Auslösendes Ereignis & Anwender hat die Software bereits zu einem früheren Zeitpunkt installiert. Sofern ein neues Update vorhanden ist, möchte er dieses installieren.\\\hline
   Vorbedingung & Richtige Entwicklungsumgebung und Software wurden bereits in einer früheren Version installiert.\\\hline
   Nachbedingung & Es sind keine Fehler aufgetreten.\\\hline
   Ergebnis & Entwicklungsumgebung und Analysesoftware sind auf dem neusten Stand für Garbage Collection Auswertungen.\\\hline
   Hauptszenario & 
	\begin{enumerate}
		\item Anwender Startet Entwicklungsumgebung
		\item Anwender sucht und findet Updates für die Analysesoftware
		\item Anwender selektiert eines oder mehrere dieser Softwarepakete
		\item Software wird aktualisiert
	\end{enumerate}
	\\\hline
   Alternativszenarien & -\\\hline
   Ausnahmeszenarien & -\\\hline
   Qualitäten & Usability\\\hline
\caption{Use-Case: Software updaten}
\end{longtable}

\begin{longtable}{|p{4cm}|p{10.5cm}|}
\hline
   \textbf{Abschnitt} & \textbf{Inhalt / Erläuterung} \\\hline
   Bezeichner & UC-03\\\hline
   Name & Garbage Collection Logdatei importieren\\\hline
   Autoren & Raffael Schmid\\\hline
   Priorität & Wichtigkeit für Systemerfolg: gross\newline Technologisches Risiko: gering\\\hline
   Kritikalität & gross\\\hline
   Verantwortlicher & Raffael Schmid\\\hline
   Kurzbeschreibung & Der Benutzer importiert die sich auf dem Dateisystem befindende Logdatei.\\\hline
   Akteure & Anwender, Logdatei Importer\\\hline
   Auslösendes Ereignis & Anwender startet Garbage Collection Log Analyse\\\hline
   Vorbedingung & Logdatei befindet sich auf dem Rechner und ist in einem der unterstützten Formate. Die Software ist vollständig installiert und gestartet.\\\hline
   Nachbedingung & Es sind keine Fehler aufgetreten. \\\hline
   Ergebnis & Die Logdatei ist in der Ansicht Logdateien ersichtlich und kann von da im Analysefenster geöffnet werden.\\\hline
   Hauptszenario & 
	\begin{enumerate}
		\item Anwender öffnet Import-Wizard
		\item Anwender navigiert zum Ordner
		\item Anwender selektiert Datei(en) und importiert diese
	\end{enumerate}
Die importierten Dateien werden gespeichert. Bei einem Neustart der Entwicklungsumgebung bleiben die zuvor importierten Dateien erhalten.
	\\\hline
   Alternativszenarien & -\\\hline
   Ausnahmeszenarien & -\\\hline
   Qualitäten & Usability\\\hline
\caption{Use-Case: Garbage Collection Logdatei importieren}
\end{longtable}

\begin{longtable}{|p{4cm}|p{10.5cm}|}
\hline
   \textbf{Abschnitt} & \textbf{Inhalt / Erläuterung} \\\hline
   Bezeichner & UC-04\\\hline
   Name & Standardauswertung anzeigen\\\hline
   Autoren & Raffael Schmid\\\hline
   Priorität & Wichtigkeit für Systemerfolg: gross\newline Technologisches Risiko: gross\\\hline
   Kritikalität & gross\\\hline
   Verantwortlicher & Raffael Schmid\\\hline
   Kurzbeschreibung & Für eine schnelle Übersicht steht eine Standard-Auswertung zur Verfügung. Diese soll eine kurze Übersicht über die Garbage Collection geben und beinhaltet zwei Charts (Heap Benutzung, Dauer Garbage Collection). \\\hline
   Akteure & Anwender, Logdatei Analyzer, Report Engine\\\hline
   Auslösendes Ereignis & Der Benutzer hat eine Garbage Collection Logdatei importiert und möchte nun die Analyse starten.\\\hline
   Vorbedingung & Bevor das Analysefenster für die Garbage Collection Logdatei geöffnet werden kann, wird die Datei eingelesen und geparst. Das heisst, dass die rohen Daten semantisch und syntaktisch analysiert und in den Arbeitsspeicher geladen werden.\\\hline
   Nachbedingung & -\\\hline
   Ergebnis & Dem Benutzer wird ein Analyse-Screen angezeigt.\\\hline
   Hauptszenario & 
	\begin{enumerate}
		\item Applikation hat die Logdatei fertig importiert.
		\item Dem Benutzer wird ein Screen mit verschiedenen Tabs angezeigt. Auf jedem Tab wird dem Benutzer eine unterschiedliche Sicht auf die Daten gezeigt.
	\end{enumerate}
	\\\hline
   Alternativszenarien & Benutzerdefinierte Auswertung\\\hline
   Ausnahmeszenarien & -\\\hline
   Qualitäten &  Korrektheit, Performance, Usability\\\hline
   Erweiterungen & UC-04.1, UC-04.2, UC-04.3, UC-04.4 \\\hline
\caption{Use-Case: Standardausertung anzeigen}
\end{longtable}

\begin{longtable}{|p{4cm}|p{10.5cm}|}
\hline
   \textbf{Abschnitt} & \textbf{Inhalt / Erläuterung} \\\hline
   Bezeichner & UC-04.1\\\hline
   Name & Anzeige Statistik Übersicht\\\hline
   Autoren & Raffael Schmid\\\hline
   Priorität & Wichtigkeit für Systemerfolg: gross\newline Technologisches Risiko: gross\\\hline
   Kritikalität & gross\\\hline
   Verantwortlicher & Raffael Schmid\\\hline
   Kurzbeschreibung & Der Analyse-Screen wurde geöffnet, dem Benutzer zeigen sich unterschiedliche Tabs. Auf dem ersten befinden sich verschiedene statistische Auswertungen der Logdatei:
   \begin{itemize}
	\item Übersicht und Grösse der verschiedenen Bereiche auf dem Heap: Initiale Grösse, endgültige Grösse
	\item Aktivitäten des Garbage Collectors: Anzahl Young Collections, Anzahl Old Collections
	\item Anzahl Garbage Collector Events (Bsp: Wechsel der Garbage Collection Strategie, etc.)
	\item Garbage Collection Zeiten (Total, Durchschnittliche, Zeit in Old Generation Garbage Collection, Prozentuale Zeit in Old Generation Garbage Collection)
	\item Durchsatz (siehe Abschnitt \ref{gc_tuning_durchsatz})
   \end{itemize}
 \\\hline
   Qualitäten &  Korrektheit, Performance, Usability\\\hline
\caption{Use-Case: Anzeige Statistik Übersicht}
\end{longtable}

\begin{longtable}{|p{4cm}|p{10.5cm}|}
\hline
   \textbf{Abschnitt} & \textbf{Inhalt / Erläuterung} \\\hline
   Bezeichner & UC-04.2\\\hline
   Name & Anzeige Heap Benutzung\\\hline
   Autoren & Raffael Schmid\\\hline
   Priorität & Wichtigkeit für Systemerfolg: gross\newline Technologisches Risiko: gross\\\hline
   Kritikalität & gross\\\hline
   Verantwortlicher & Raffael Schmid\\\hline
   Kurzbeschreibung & Die Heap Usage (Heap Benutzung) zeigt dem Benutzer anhand einer Grafik, zu welchem Zeitpunkt wieviel Speicher des Heaps verwendet wurde. Zusätzlich werden die Zeitpunkte inklusive entsprechendem Typ (Old- / Young-Collection) der Garbage Collection angezeigt.  \\\hline
   Qualitäten & Korrektheit, Performance, Usability\\\hline
\caption{Use-Case: Anzeige Heap Benutzung}
\end{longtable}

\begin{longtable}{|p{4cm}|p{10.5cm}|}
\hline
   \textbf{Abschnitt} & \textbf{Inhalt / Erläuterung} \\\hline
   Bezeichner & UC-04.3\\\hline
   Name & Anzeige Dauer Garbage Collection\\\hline
   Autoren & Raffael Schmid\\\hline
   Priorität & Wichtigkeit für Systemerfolg: mittel\newline Technologisches Risiko: mittel\\\hline
   Kritikalität & Mittel\\\hline
   Verantwortlicher & Raffael Schmid\\\hline
   Kurzbeschreibung & Für jede Garbage Collection ist innerhalb der Logdatei einen Eintrag mit den Informationen, wie viel Speicher von toten Objekten befreit wurde und wie lange die Collection gedauert hat. In diesem Report geht es um die Darstellung dieser Daten.\\\hline
   Qualitäten &  Korrektheit, Performance, Usability\\\hline
\caption{Use-Case: Anzeige Dauer Garbage Collection}
\end{longtable}

\begin{longtable}{|p{4cm}|p{10.5cm}|}
\hline
   \textbf{Abschnitt} & \textbf{Inhalt / Erläuterung} \\\hline
   Bezeichner & UC-05\\\hline
   Name &Profil (benutzerdefinierte Auswertung) erstellen\\\hline
   Autoren & Raffael Schmid\\\hline
   Priorität & Wichtigkeit für Systemerfolg: niedrig\newline Technologisches Risiko: niedrig\\\hline
   Kritikalität & Niedrig\\\hline
   Verantwortlicher & Raffael Schmid\\\hline
   Kurzbeschreibung & Der Benutzer kann ein eigenes Profil erstellen. Dem Profil können eigene, benutzerdefinierte Charts hinzugefügt werden. Auf jedem Chart können unterschiedliche Serien\footnote{Eine Serie definiert welche Daten auf der X- respektive Y-Achse angezeigt werden sollen.} hinzugefügt werden. Die Profile sind persistent und können exportiert wie auch importiert werden.\\\hline
   Akteure & Anwender, Logdatei Analyzer, Report Engine\\\hline
   Auslösendes Ereignis & Die Applikation hat die Datei fertig eingelesen und geparst.\\\hline
   Vorbedingung & Die Logdatei wurde ohne Fehler eingelesen und befindet sich im strukturierten Format im Arbeitsspeicher.\\\hline
   Nachbedingung & -\\\hline
   Ergebnis & Dem Benutzer wird ein benutzerdefiniertes Analysefenster angezeigt.\\\hline
   Hauptszenario & Unabhängig vom Zyklus der Garbage Collection Analyse, kann der Benutzer ein eigenes Profil erstellen. Ein Profil besteht initial aus einem Übersichtsfenster der Garbage Collection und kann um eigene, benutzerdefinierte Charts erweitert werden. Sollte die Entwicklungsumgebung mit der Analysesoftware geschlossen werden, bleiben die erstellten Profile erhalten.\\\hline
   Alternativszenarien & -\\\hline
   Ausnahmeszenarien & -\\\hline
   Qualitäten & Korrektheit \\\hline
\caption{Use-Case: Profil (benutzerdefinierte Auswertung) erstellen }
\end{longtable}




\begin{longtable}{|p{4cm}|p{10.5cm}|}
\hline
   \textbf{Abschnitt} & \textbf{Inhalt / Erläuterung} \\\hline
   Bezeichner & UC-6\\\hline
   Name & Hilfesystem\\\hline
   Autoren & Raffael Schmid\\\hline
   Priorität & Wichtigkeit für Systemerfolg: niedrig\newline Technologisches Risiko: mittel\\\hline
   Kritikalität & Mittel\\\hline
   Verantwortlicher & Raffael Schmid\\\hline
   Kurzbeschreibung & Dem Benutzer steht eine eine indexbasierte\footnote{Generelle Hilfe mit Informationen zur Garbage Collection,Vorgehensweise bei Performance Problemen, alternative Werkzeuge, etc.} und eine kontextsensitive\footnote{Hilfe zur aktuellen View oder Aktion des Benutzers} Hilfe zur Verfügung. \\\hline
   Akteure & Anwender\\\hline
   Auslösendes Ereignis & Anwender hat Plugin installiert, weiss nicht wie eine Analyse gestartet werden kann.\\\hline
   Vorbedingung & Entwicklungsumgebung und Software sind installiert.\\\hline
   Nachbedingung & -\\\hline
   Ergebnis & Anwender kennt Software\\\hline
   Hauptszenario &	\begin{itemize}
		\item \textbf{Indexbasierte Hilfe: } Der Benutzer kennt sich im Thema Garbage Collection und auf der Analysesoftware noch nicht aus. Er holt sich Hilfe über die indexbasierte Hilfe. 
		\item \textbf{Kontextsensitive Hilfe: } Der Benutzer befindet sich in einem Fenster oder möchte eine Aktion ausführen (Context), das Hilfesystem zeigt ihm dazu die notwendingen Informationen.
	\end{itemize}
	\\\hline
   Alternativszenarien & -\\\hline
   Ausnahmeszenarien & -\\\hline
   Qualitäten & Internationalisierung, Usability\\\hline
\caption{Use-Case: Hilfesystem}
\end{longtable}


\subsection{Use Cases Entwickler}
\begin{longtable}{|p{4cm}|p{10.5cm}|}
  \caption{Use-Case: Software installieren}\\\hline
   \textbf{Abschnitt} & \textbf{Inhalt / Erläuterung} \\\hline
   Bezeichner & UC-A-01\\\hline
   Name & Software installieren\\\hline
   Autoren & Raffael Schmid\\\hline
   Priorität & Wichtigkeit für Systemerfolg: hoch\newline Technologisches Risiko: mittel\\\hline
   Kritikalität & Hoch\\\hline
   Quelle & Raffael Schmid\\\hline
   Verantwortlicher & Raffael Schmid\\\hline
   Kurzbeschreibung & Der Benutzer kann das Plugin\footnote{mit Plugin ist die Garbage Collection Analysesoftware gemeint} in seiner Entwicklungsumgebung via Eingabe einer Update-Seite installieren.\\\hline
   Auslösendes Ereignis & Anwender möchte die Auswertung einer Garbage Collection Log Datei machen.\\\hline
   Akteure & Anwender\footnote{Anwender kann sowohl Performance Engineer oder Java Entwickler sein}, Update-Server\\\hline
   Vorbedingung & Richtige Entwicklungsumgebung ist bereits installiert. Anwender ist noch nicht im Besitz des Plugins.\\\hline
   Nachbedingung & Anwender hat Plugin fertig installiert.\\\hline
   Ergebnis & Entwicklungsumgebung ist bereit für Garbage Collection Auswertungen der JRockit Garbage Collection Log Dateien.\\\hline
   Hauptszenario & 
         \begin{enumerate}
		\item Anwender Startet Entwicklungsumgebung
		\item Anwender gibt Update-Seite an
		\item Anwender selektiert zu installierendes Packet
		\item Plugin wird installiert	
 	\end{enumerate}
	\\\hline
   Alternativszenarien & Anwender installiert Plugin aus lokalem Dateisystem\\\hline
   Ausnahmeszenarien & -\\\hline
   Qualitäten & QA-01 (Installierbarkeit)\\\hline
\end{longtable}

\begin{longtable}{|p{4cm}|p{10.5cm}|}
\caption{Use-Case: Software updaten}\\\hline
   \textbf{Abschnitt} & \textbf{Inhalt / Erläuterung} \\\hline
   Bezeichner & UC-A-02\\\hline
   Name & Software updaten\\\hline
   Autoren & Raffael Schmid\\\hline
   Priorität & Wichtigkeit für Systemerfolg: hoch\newline Technologisches Risiko: mittel\\\hline
   Kritikalität & Mittel\\\hline
   Quelle & Raffael Schmid\\\hline
   Verantwortlicher & Raffael Schmid\\\hline
   Kurzbeschreibung & Der Benutzer kann das Plugin updaten.\\\hline
   Auslösendes Ereignis & Anwender hat das Plugin bereis zu einem früheren Zeitpunkt installiert. Sofern ein neues Update dieses Plugins vorhanden ist, möchte er dieses installieren.\\\hline
   Akteure & Anwender, Update-Server\\\hline
   Vorbedingung & Richtige Entwicklungsumgebung und Garbage Collection Log Analysis Plugin ist bereits in einer früheren Version installiert.\\\hline
   Nachbedingung & Plugin ist auf dem neusten Stand.\\\hline
   Ergebnis & Entwicklungsumgebung und Plugin sind auf dem neusten Stand für Garbage Collection Auswertungen.\\\hline
   Hauptszenario & 
	\begin{enumerate}
		\item Anwender Startet Entwicklungsumgebung
		\item Anwender sucht nach neuen Updates
		\item Anwendung findet neue Updates
		\item Anwender selektiert Packete die er updaten möchte
		\item Plugin wird aktualisiert
	\end{enumerate}
	\\\hline
   Alternativszenarien & Anwender updated Plugin aus lokalem Dateisystem\\\hline
   Ausnahmeszenarien & -\\\hline
   Qualitäten & QA-01 (Installierbarkeit)\\\hline
\end{longtable}

\begin{longtable}{|p{4cm}|p{10.5cm}|}
\caption{Use-Case: Garbage Collection Log Datei importieren}\\\hline
   \textbf{Abschnitt} & \textbf{Inhalt / Erläuterung} \\\hline
   Bezeichner & UC-A-03\\\hline
   Name & Garbage Collection Log Datei importieren\\\hline
   Autoren & Raffael Schmid\\\hline
   Priorität & Wichtigkeit für Systemerfolg: hoch\newline Technologisches Risiko: hoch\\\hline
   Kritikalität & Hoch\\\hline
   Quelle & Raffael Schmid\\\hline
   Verantwortlicher & Raffael Schmid\\\hline
   Kurzbeschreibung & Der Benutzer kann eine sich auf seinem Rechner befindende Garbage Collection Log Datei importierten.\\\hline
   Auslösendes Ereignis & Anwender möchte Auswertung starten.\\\hline
   Akteure & Anwender, Log Datei Prozessor\\\hline
   Vorbedingung & Log Datei befindet sich auf dem Rechner und ist in einem der unterstützten Formate. Garbage Collection Log Analysis Plugin ist vollständig installiert. Entwicklungsumgebung ist gestartet.\\\hline
   Nachbedingung & \\\hline
   Ergebnis & Die Log Datei befindet sich im Memory und steht für die Auswertung (Charts, Reports) dem Benutzer zur Verfügung.\\\hline
   Hauptszenario & 
	\begin{enumerate}
		\item Anwender öffnet Import-Dialog
		\item Anwender navigiert zur Datei
		\item Anwender startet Import-Prozess
	\end{enumerate}
	\\\hline
   Alternativszenarien & -\\\hline
   Ausnahmeszenarien & -\\\hline
   Qualitäten & -\\\hline
\end{longtable}

\begin{longtable}{|p{4cm}|p{10.5cm}|}
\caption{Use-Case: Garbage Collection Log Datei analysieren}\\\hline
   \textbf{Abschnitt} & \textbf{Inhalt / Erläuterung} \\\hline
   Bezeichner & UC-A-04\\\hline
   Name & Garbage Collection Log Datei analysieren\\\hline
   Autoren & Raffael Schmid\\\hline
   Priorität & Wichtigkeit für Systemerfolg: hoch\newline Technologisches Risiko: hoch\\\hline
   Kritikalität & Hoch\\\hline
   Quelle & Raffael Schmid\\\hline
   Verantwortlicher & Raffael Schmid\\\hline
   Kurzbeschreibung & Der Benutzer kann die importierte Datei nun weiter analysieren.\\\hline
   Auslösendes Ereignis & Die Applikation hat die Datei fertig importiert, der Benutzer öffnet eine Datei die bereits importiert wurde.\\\hline
   Akteure & Anwender, Log Datei Analyzer, Report Engine\\\hline
   Vorbedingung & Die Log Datei wurde ohne Fehler eingelesen und befindet sich nun im Arbeitsspeicher.\\\hline
   Nachbedingung & -\\\hline
   Ergebnis & Dem Benutzer wird ein Analsye-Screen angezeigt.\\\hline
   Hauptszenario & 
	\begin{enumerate}
		\item Applikation hat die Log-Datei fertig importiert oder Benutzer öffnet Datei, die sich bereits im Arbeitsspeicher befindet.
		\item Dem Benutzer wird ein Screen mit verschiedenen Tabs angezeigt. Auf jedem Tab wird dem Benutzer eine unterschiedliche Sicht auf die Daten gezeigt.
	\end{enumerate}
	\\\hline
   Alternativszenarien & -\\\hline
   Ausnahmeszenarien & -\\\hline
   Qualitäten & -\\\hline
   Erweiterungen & UC-A-04-a (Statistik Übersicht), UC-A-04-b (Heap Benutzung), UC-A-04-c (Dauer Garbage Collection), UC-A-04-d (Benutzerdefiniert Auswertung) \\\hline
\end{longtable}

\begin{longtable}{|p{4cm}|p{10.5cm}|}
\caption{Use-Case: Statistik Übersicht}\\\hline
   \textbf{Abschnitt} & \textbf{Inhalt / Erläuterung} \\\hline
   Bezeichner & UC-A-04-a\\\hline
   Name & Statistik Übersicht\\\hline
   Autoren & Raffael Schmid\\\hline
   Priorität & Wichtigkeit für Systemerfolg: hoch\newline Technologisches Risiko: hoch\\\hline
   Kritikalität & Hoch\\\hline
   Quelle & Raffael Schmid\\\hline
   Verantwortlicher & Raffael Schmid\\\hline
   Kurzbeschreibung & Der Analyse-Screen wurde geöffnet, dem Benutzer zeigen sich unterschiedliche Tabs. Auf dem ersten befinden sich verschiedene statistische Auswertungen der Log Datei:
   \begin{itemize}
	\item Übersicht und Grösse der verschiedenen Bereiche auf dem Heap: Initiale Grösse, endgültige Grösse
	\item Aktivitäten des Garbage Collectors: Anzahl Young Collections, Anzahl Old Collections
	\item Anzahl Garbage Collector Events (Bsp: Change Garbage Collector Strategy, etc.)
	\item Garbage Collection Zeiten (Total, Durchschnittliche, Zeit in Old Generation Garbage Collection, Prozentuale Zeit in Old Generation Garbage Collection)
   \end{itemize}
 \\\hline
   Qualitäten & QA-06 (Optimale Wartezeiten)\\\hline
\end{longtable}

\begin{longtable}{|p{4cm}|p{10.5cm}|}
\caption{Use-Case: Heap Benutzung}\\\hline
   \textbf{Abschnitt} & \textbf{Inhalt / Erläuterung} \\\hline
   Bezeichner & UC-A-04b\\\hline
   Name & Heap Benutzung\\\hline
   Autoren & Raffael Schmid\\\hline
   Priorität & Wichtigkeit für Systemerfolg: hoch\newline Technologisches Risiko: hoch\\\hline
   Kritikalität & Hoch\\\hline
   Quelle & Raffael Schmid\\\hline
   Verantwortlicher & Raffael Schmid\\\hline
   Kurzbeschreibung & Die Heap Usage (Heap Benutzung) zeigt dem Benutzer anhand einer Grafik, zu welchem Zeitpunkt wieviel Speicher des Heaps verwendet wurde. Zusätzlich werden die Zeitpunkte inklusive entsprechendem Typ (Old- / Young-Collection) der Garbage Collection angezeigt.  \\\hline
   Qualitäten & QA-06 (Optimale Wartezeiten)\\\hline
\end{longtable}

\begin{longtable}{|p{4cm}|p{10.5cm}|}
\caption{Use-Case: Dauer Garbage Collection}\\\hline
   \textbf{Abschnitt} & \textbf{Inhalt / Erläuterung} \\\hline
   Bezeichner & UC-A-04c\\\hline
   Name & Dauer Garbage Collection\\\hline
   Autoren & Raffael Schmid\\\hline
   Priorität & Wichtigkeit für Systemerfolg: mittel\newline Technologisches Risiko: mittel\\\hline
   Kritikalität & Mittel\\\hline
   Quelle & Raffael Schmid\\\hline
   Verantwortlicher & Raffael Schmid\\\hline
   Kurzbeschreibung & Für jede Garbage Collection ist innerhalb der Log Datei einen Eintrag mit den Informationen, wie viel Speicher von toten Objekten befreit wurde und wie lange die Collection gedauert hat. In diesem Report geht es um die Darstellung dieser Daten.\\\hline
   Qualitäten & QA-06 (Optimale Wartezeiten)\\\hline
\end{longtable}

\begin{longtable}{|p{4cm}|p{10.5cm}|}
\caption{Use-Case: Benutzerdefiniert Auswertung }\\\hline
   \textbf{Abschnitt} & \textbf{Inhalt / Erläuterung} \\\hline
   Bezeichner & UC-A-04d\\\hline
   Name & Benutzerdefiniert Auswertung\\\hline
   Autoren & Raffael Schmid\\\hline
   Priorität & Wichtigkeit für Systemerfolg: niedrig\newline Technologisches Risiko: niedrig\\\hline
   Kritikalität & Niedrig\\\hline
   Quelle & Raffael Schmid\\\hline
   Verantwortlicher & Raffael Schmid\\\hline
   Kurzbeschreibung & Der Benutzer hat hier die Möglichkeit, von den verfügbaren Daten für X- und Y-Achse einen Messtyp auszuwählen und damit einen eigenen Chart zu definieren. \\\hline
   Qualitäten & QA-06 (Optimale Wartezeiten)\\\hline
   Erweiterungen & UC-A-04d1 (Auswahl administrieren)\\\hline
\end{longtable}

\begin{longtable}{|p{4cm}|p{10.5cm}|}
\caption{Use-Case: Benutzerdefinierte Auswertung speichern, löschen }\\\hline
   \textbf{Abschnitt} & \textbf{Inhalt / Erläuterung} \\\hline
   Bezeichner & UC-A-04d1\\\hline
   Name & Benutzerdefinierte Auswertung speichern, löschen\\\hline
   Autoren & Raffael Schmid\\\hline
   Priorität & Wichtigkeit für Systemerfolg: niedrig\newline Technologisches Risiko: niedrig\\\hline
   Kritikalität & Niedrig\\\hline
   Quelle & Raffael Schmid\\\hline
   Verantwortlicher & Raffael Schmid\\\hline
   Kurzbeschreibung & Die vom Anwender definierten Charts können lokal gespeichert werden. Die gespeicherten Daten sind Name, optionale Beschreibung, Datentyp X-Achse, Datentyp Y-Achse.\\\hline
   Qualitäten & -\\\hline
\end{longtable}

\subsection{Use Cases Entwickler}
\begin{longtable}{|p{4cm}|p{10.5cm}|}
\caption{Use-Case: Neuer Software-Release erstellen}\\\hline
   \textbf{Abschnitt} & \textbf{Inhalt / Erläuterung} \\\hline
   Bezeichner & UC-E-01\\\hline
   Name & Neuer Software-Release erstellen\\\hline
   Autoren & Raffael Schmid\\\hline
   Priorität & Wichtigkeit für Systemerfolg: hoch\newline Technologisches Risiko: gering\\\hline
   Kritikalität & Mittel\\\hline
   Quelle & Raffael Schmid\\\hline
   Verantwortlicher & Raffael Schmid\\\hline
   Kurzbeschreibung & Der Entwickler kann neue Releases des Plugins erstellen und diese via Update-Seite verfügbar machen.\\\hline
   Auslösendes Ereignis & Entwickler hat einen neuen Bug gefixt oder ein Feature implementiert.\\\hline
   Akteure & Entwickler\\\hline
   Vorbedingung & Entwicklung wurde abgeschlossen, Tests sind durchgelaufen.\\\hline
   Nachbedingung & -\\\hline
   Ergebnis & Neues Update ist via Update-Seite verfügbar und kann vom Anwender darüber installiert werden.\\\hline
   Hauptszenario & 
	\begin{enumerate}
		\item Entwickler entwickelt Feature oder macht Bug-Fix.
		\item Entwickler startet den automatischen Build-Deploy der Software
	\end{enumerate}
	\\\hline
   Alternativszenarien & -\\\hline
   Ausnahmeszenarien & Im Fehlerfall wird der Entwickler benachrichtigt, das neue Softwarepaket wird nicht via Update-Seite bereitgestellt.\\\hline
   Qualitäten & - \\\hline
\end{longtable}





