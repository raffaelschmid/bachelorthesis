\chapter{Informationen}

\section*{Inhalt Datenträger}
  \begin{longtable}{|p{5.5cm}|p{5cm}|p{2.5cm}|}
\hline
  \textbf{Pfad} & \textbf{Inhalt} & \textbf{Format}\\\hline
    Dokumentation &Beinhaltet alle Dokumente, die im Zusammenhang mit der Bachelorthesis entstanden sind.
&Pdf, Word, PowerPoint\\\hline
    Ressourcen & Beinhaltet Dokumente die im Zusammenhang mit der Bachelorthesis hilfreich waren. & Pdf\\\hline
    Software/development/data & Beinhaltet den Quelltext der Analysesoftware. & Java Projekt\\\hline
    Software/sampling/data & Beinhaltet ein Programm zur Generierung von Garbage Collection Logdateien. Einige Beispiele solcher Dateien sind ebenfalls vorhanden. & Java Projekt\\\hline
      \caption{Inhalt Datenträger}\\
  \end{longtable}

\section*{Repository}
Der Quelltext und alle Dokumente befinden sich auf Github, einem öffentlich verfügbaren Git-Repository. Mittels folgendem Kommando\footnote{Dafür ist die Installation der Software Git erforderlich.} kann alles aus dem Repository ausgecheckt werden.

\begin{lstlisting}[caption=Checkout Quelltext Repository]
git clone git://github.com/schmidic/bachelorthesis.git
\end{lstlisting}



