\chapter{Informationen}

\section{Inhalt Datenträger}
  \begin{longtable}{|p{5cm}|p{8cm}|}
\hline
  \textbf{Pfad} & \textbf{Inhalt}\\\hline
    Dokumentation & Beinhaltet alle Dokumente, die im Zusammenhang mit der Bachelorthesis entstanden sind.\\\hline
    Ressourcen & Beinhaltet Dokumente die im Zusammenhang mit der Bachelorthesis hilfreich waren.\\\hline
    Software/development/data & Beinhaltet alle Plugin-Projekte aus welcher die Garbage Collection Analysesoftware besteht.\\\hline
    Software/sampling/data & Das sich darin befindende Projekt beinhaltet Garbage Collection Logs die mittels unterschiedlichen Einstellungen und den ebenfalls enthaltenen Klassen entstanden sind. \\\hline
      \caption{Inhalt Datenträger}\\
  \end{longtable}

\section{Quelltext Repository}
Der Quelltext und die Dokumentation befinden sich zusätzlich auf Github in einem öffentlich verfügbaren Git-Repository. Mittels folgendem Kommando\footnote{Dafür ist die Installation der Software Git erforderlich.} kann der Quelltext aus dem Repository ausgecheckt werden.

\begin{lstlisting}[caption=Checkout Quelltext Repository]
git clone git://github.com/schmidic/bachelorthesis.git
\end{lstlisting}



