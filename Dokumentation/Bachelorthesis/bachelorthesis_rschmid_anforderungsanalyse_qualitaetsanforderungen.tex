\subsection{Qualitätsanforderungen}
\begin{longtable}{|p{4cm}|p{10.5cm}|}
    \caption{Qualitätsanforderung: Installierbarkeit}\\\hline
   \textbf{Abschnitt} & \textbf{Inhalt / Erläuterung} \\\hline
   Bezeichner & QA-01\\\hline
   Name & Installierbarkeit\\\hline
   Autoren & Raffael Schmid\\\hline
   Priorität & Wichtigkeit für Systemerfolg: hoch\newline Technologisches Risiko: mittel\\\hline
   Kritikalität & Klein\\\hline
   Quelle & Raffael Schmid\\\hline
   Verantwortlicher & Raffael Schmid\\\hline
   Kurzbeschreibung & Der Benutzer kann die Software in kurzer Zeit installieren. In diesem Zusammenhang ist es wichtig, dass das zu installierende Softwarepaket möglichst klein ist.\\\hline
\end{longtable}

\begin{longtable}{|p{4cm}|p{10.5cm}|}
    \caption{Qualitätsanforderung: Erweiterbarkeit}\\\hline
   \textbf{Abschnitt} & \textbf{Inhalt / Erläuterung} \\\hline
   Bezeichner & QA-02\\\hline
   Name & Erweiterbarkeit\\\hline
   Autoren & Raffael Schmid\\\hline
   Priorität & Wichtigkeit für Systemerfolg: gering\newline Technologisches Risiko: gering\\\hline
   Kritikalität & Klein\\\hline
   Quelle & Raffael Schmid\\\hline
   Verantwortlicher & Raffael Schmid\\\hline
   Kurzbeschreibung & Nebst der Analyse von Garbage Collection Log Dateien für die JRockit Virtual Machine sollen auch andere Log-Formate integriert werden. Der Kern der Applikation soll dabei möglichst schlank sein, die zusätzliche Gewünschte Auswertung soll als eigenes Feature installierbar sein.\\\hline
\end{longtable}

\begin{longtable}{|p{4cm}|p{10.5cm}|}
    \caption{Qualitätsanforderung: Testabdeckung}\\\hline
   \textbf{Abschnitt} & \textbf{Inhalt / Erläuterung} \\\hline
   Bezeichner & QA-03\\\hline
   Name & Testabdeckung\\\hline
   Autoren & Raffael Schmid\\\hline
   Priorität & Wichtigkeit für Systemerfolg: mittel\newline Technologisches Risiko: gering\\\hline
   Kritikalität & Mittel\\\hline
   Quelle & Raffael Schmid\\\hline
   Verantwortlicher & Raffael Schmid\\\hline
   Kurzbeschreibung & Um den langfristigen Erfolg dieser Software zu gewährleisten muss eine entsprechende Testabdeckung vorhanden sein - dies um insbesondere die Regression zu vermeiden.\\\hline
\end{longtable}

\begin{longtable}{|p{4cm}|p{10.5cm}|}
    \caption{Qualitätsanforderung: Plattformunabhängigkeit}\\\hline
   \textbf{Abschnitt} & \textbf{Inhalt / Erläuterung} \\\hline
   Bezeichner & QA-04\\\hline
   Name & Plattformunabhängigkeit\\\hline
   Autoren & Raffael Schmid\\\hline
   Priorität & Wichtigkeit für Systemerfolg: mittel\newline Technologisches Risiko: mittel\\\hline
   Kritikalität & Mittel\\\hline
   Quelle & Raffael Schmid\\\hline
   Verantwortlicher & Raffael Schmid\\\hline
   Kurzbeschreibung & Die Software respektive das Plugin soll so implementiert sein, dass es auf unterschiedlichen Systemen (Betriebssystemen) lauffähig ist.\\\hline
\end{longtable}

\begin{longtable}{|p{4cm}|p{10.5cm}|}
    \caption{Qualitätsanforderung: Internationalisierung}\\\hline
   \textbf{Abschnitt} & \textbf{Inhalt / Erläuterung} \\\hline
   Bezeichner & QA-05\\\hline
   Name & Internationalisierung\\\hline
   Autoren & Raffael Schmid\\\hline
   Priorität & Wichtigkeit für Systemerfolg: gering\newline Technologisches Risiko: gering\\\hline
   Kritikalität & Mittel\\\hline
   Quelle & Raffael Schmid\\\hline
   Verantwortlicher & Raffael Schmid\\\hline
   Kurzbeschreibung & Es soll die Möglichkeit bestehen, als Erweiterung zusätzliche Sprachpakete zu installieren.\\\hline
\end{longtable}

\begin{longtable}{|p{4cm}|p{10.5cm}|}
    \caption{Qualitätsanforderung: Optimale Wartezeiten}\\\hline
   \textbf{Abschnitt} & \textbf{Inhalt / Erläuterung} \\\hline
   Bezeichner & QA-06\\\hline
   Name & Optimale Wartezeiten\\\hline
   Autoren & Raffael Schmid\\\hline
   Priorität & Wichtigkeit für Systemerfolg: hoch\newline Technologisches Risiko: mittel\\\hline
   Kritikalität & Mittel\\\hline
   Quelle & Raffael Schmid\\\hline
   Verantwortlicher & Raffael Schmid\\\hline
   Kurzbeschreibung & Aufgrund der Masse an zu verarbeitenden Daten ist zu erwarten, dass es an einigen Orten Wartezeiten entstehen. Diese sollten allerdings hinsichtlich Benutzbarkeit optimiert sein. Beispiel: Lazy Loading von nicht aktiven Tabs, etc.\\\hline
\end{longtable}

\begin{longtable}{|p{4cm}|p{10.5cm}|}
    \caption{Qualitätsanforderung: Hilfesystem}\\\hline
   \textbf{Abschnitt} & \textbf{Inhalt / Erläuterung} \\\hline
   Bezeichner & QA-07\\\hline
   Name & Hilfesystem\\\hline
   Autoren & Raffael Schmid\\\hline
   Priorität & Wichtigkeit für Systemerfolg: gering\newline Technologisches Risiko: gering\\\hline
   Kritikalität & Gering\\\hline
   Quelle & Raffael Schmid\\\hline
   Verantwortlicher & Raffael Schmid\\\hline
   Kurzbeschreibung & Dem Benutzer werden zwei Hilfesysteme zur Verfügung gestellt: 
	\begin{itemize}
		\item Generelle Hilfe mit Informationen zur Garbage Collection,Vorgehensweise bei Performance Problemen, alternative Werkzeuge, etc. 
		\item Context-Sensitive Hilfe zur aktuellen View oder Aktion des Benutzers
	\end{itemize}\\\hline
\end{longtable}

