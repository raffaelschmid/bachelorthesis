\chapter*{Glossar}\label{glossar}
\addcontentsline{toc}{chapter}{Glossar}
  \begin{longtable}{|p{5cm}|p{6cm}|p{3cm}|}
      \caption{Glossar}\\
\hline
  \textbf{Wort} & \textbf{Beschreibung} & \textbf{Herkunft}\\\hline
  Continuous Integration  &&\\\hline
  Bundle (Eclipse) & siehe Plugin & siehe Plugin\\\hline
  Category (Eclipse) & Auf einer Eclipse Update-Seite werden Features zur Verfügung gestellt. Diese können logisch noch einmal in Kategorien unterteilt werden. & Eclipse\\\hline
  Feature (Eclipse) & Als Feature verpackt man im Eclipse-Umfeld eine logische Einheit an Funktionalität. & Eclipse\\\hline
  Fragment & Manchmal macht es im Eclipse-Umfeld Sinn, gewisse Teile der Applikation optional zu definieren. In diesem Fall verwendet man Fragmente. Sie erlauben die Erweiterung eines bestehenden Bundles (Host-Bundle) und haben Zugriff auf alle auch nicht exportierten Packete. Test-Projekte werden oft auch als Fragmente definiert, da ihnen der volle Zugriff auf das Host-Bundle gewährleistet wird. & Eclipse\\\hline
  Garbage Collection & & \\\hline
  Garbage Collection Algorithmus, Strategie & & \\\hline
  Plugin (Eclipse) & Ein Plugin ist eine technische Trennung von gewissen logischen Softwareteilen. Dabei definiert man die Schnittstelle zu anderen Plugins mittels einer Manifest.mf respektive einer plugin.xml Datei. Diese definiert die Abhängigkeiten (Import, Required Bundles inklusive den jeweiligen Versionen), die Exportierten Klassen und die Extension Points  & Eclipse\\\hline
  Update Site & Eine Update Site ist eine per Eclipse-Konvention aufgebaute Webseite die via das Http-Protokoll zugänglich ist und Software-Features enthält.& Eclipse\\\hline
  Virtual Machine (JRockit, HotSpot) & & \\\hline
  \end{longtable}


