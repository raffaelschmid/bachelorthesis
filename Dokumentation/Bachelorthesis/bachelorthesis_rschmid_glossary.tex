\chapter{Glossar}\label{glossar}
  \begin{longtable}{|p{4.5cm}|p{6cm}|p{4.1cm}|}
\hline
  \textbf{Wort} & \textbf{Beschreibung} & \textbf{Herkunft}\\\hline
  Continuous Integration  & Kontinuierliches Kompilieren, Bilden und Testen einer Applikation. Hilft einem oder mehreren Entwicklern eine bessere Softwarequalität zu erreichen.&-\\\hline
  Bundle (Eclipse) & siehe Plugin & siehe Plugin\\\hline
  Category (Eclipse) & Auf einer Eclipse Update-Seite werden Features zur Verfügung gestellt. Diese können logisch noch einmal in Kategorien unterteilt werden. & Eclipse\\\hline
  Feature (Eclipse) & Als Feature verpackt man im Eclipse-Umfeld eine logische Einheit an Funktionalität. & Eclipse\\\hline
  Fragment & Manchmal macht es im Eclipse-Umfeld Sinn, gewisse Teile der Applikation optional zu definieren. In diesem Fall verwendet man Fragmente. Sie erlauben die Erweiterung eines bestehenden Bundles (\textit{Host-Bundle}) und haben Zugriff auf alle auch nicht exportierten Packete dieses Host-Bundles. Test-Projekte werden oft auch als Fragmente definiert, da ihnen der volle Zugriff auf das \textit{Host-Bundle} gewährleistet wird. & Eclipse\\\hline
  Data Binding & Data Binding ist ein Mechanismus in Client Applikationen, um die Werte eines Bedienelementes mit dem hinterlegten View-Model zu verbinden. Änderungen am Wert des Eingabefeldes werden beispielsweise an das Model propagiert - sofern es sich um ein bidirektionales Binding handelt, auch umgekehrt.  & - \\\hline
  Old Collection & Eine Old Collection bezeichnet die Garbage Collection auf der Old Generation. & Speichermanagement\\\hline
  Old Generation & Bei einigen Garbage Collection Algorithmen wird der Heap in Generationen unterteilt. Der Bereich mit den jungen Objekten wird Old Generation genannt.  & Speichermanagement \\\hline
  Old Space &  siehe Old Generation & siehe Old Generation \\\hline
  Parseprozess & Bezeichnet die Interpretation und Aufbereitung von unstrukturierten Daten in eine strukturierte Form (Domänenmodell, Liste, Key-Value Datenstruktur, etc.). & Compilerbau \\\hline
  Plugin (Eclipse) & Ein Plugin ist eine technische Trennung von gewissen logischen Softwareteilen. Dabei definiert man die Schnittstelle zu anderen Plugins mittels einer Manifest.mf respektive einer plugin.xml Datei. Diese definiert die Abhängigkeiten (Import, Required Bundles inklusive den jeweiligen Versionen), die Exportierten Klassen und die Extension Points  & Eclipse\\\hline
Rich Client Framework & Software (-paket) mittels welchem sich Rich Client Applikationen entwickeln lassen.\\\hline
Rich Client Plattform & Der Begriff Rich Client Plattform wird in diesem Dokument als Synonym für Desktop- respektive Client-Applikation verwendet. & Vor gut 10 Jahren verlagerte sich die Logik vom Client auf den Server. Jede Interaktion mit dem System fand über eine Verbindung zum Server statt, der Client stellte die Inhalte nur dar. Die Anwendungen waren wenig benutzerfreundlich. Es gab deshalb eine Gegenbewegung zu den Rich Client Applikationen, dabei wird mindestens ein Teil der Logik wieder in den Client verlagert.\\\hline
  Update Site & Eine Update Site ist eine per Eclipse-Konvention aufgebaute Webseite die via das Http-Protokoll zugänglich ist und Software-Features enthält.& Eclipse\\\hline
  Virtual Machine (JRockit, HotSpot) & Laufzeitumgebung die beispielsweise eine Java Applikation ausführt. & Softwareentwicklung  \\\hline
  Wizard & Dialog innerhalb einer Anwendung, der den Benutzer durch einen Prozess führt. & -\\\hline
  Young Collection & Eine Young Collection bezeichnet die Garbage Collection auf der Young Generation.  & Speichermanagement\\\hline
  Young Generation &  Bei einigen Garbage Collection Algorithmen wird der Heap in Generationen unterteilt. Der Bereich mit den jungen Objekten wird Young Generation genannt.  & Speichermanagement \\\hline
  Young Space &  siehe Young Generation & siehe Young Generation \\\hline
User Interface &  Als User Interface (Benutzeroberfläche) werden alle Schnittstellen zwischen Mensch und Maschine bezeichnet. Diese können textbasiert oder grafisch sein (GUI). &  \\\hline
      \caption{Glossar}\\
  \end{longtable}


