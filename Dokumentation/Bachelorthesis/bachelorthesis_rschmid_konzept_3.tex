\chapter{Konzept (Infrastruktur)}\label{konzept_3}
\section{Verwendete Werkzeuge}
\subsection{Build-Automatisierung}
Die Automatisierung des Software-Builds ist hinsichtlich der Integration in ein Continuous Integration System wichtig. Zusätzlich entfallen so zeitaufwändige Tasks wie die Paketierung und das Deployment der Applikation.
Als Werkzeug zum automatisierten Build der Software wird Maven Tycho\footnote{Im Bereich der Eclipse Rich Client Entwicklung kann entweder PDE Build, ein auf Apache Ant basiertes Build-System für Eclipse RCP Applikationen\cite{vogelZapfPdeBuild} oder die Maven-Integration Tycho (http://tycho.sonatype.org) verwendet werden.} verwendet.

Tycho ist relativ neu und bringt im Vergleich mit dem PDE Build einige Vorteile mit sich:
\begin{itemize}
	\item Maven folgt dem Prinzip ``Convention over Configuration''\footnote{Das Prinzip ``Convention over Configuration'' hat zur Folge, dass im Wesentlichen nur von den Standardeinstellungen abweichende Werte konfiguriert werden müssen.} - die Konfiguration des Builds wird dadurch wesentlich einfacher.
	\item Maven ist de facto Standard bei den Build-Werkzeugen.
\end{itemize}

\section{Issue Tracker}\label{issue_tracker}
Als Issue Tracker wird Jira verwendet. Es handelt sich dabei um eine kostenpflichtige aber relativ günstige Software für das Issue-Tracking.

\section{Versionskontrolle}
Zur Versionsverwaltungssysteme kommen mehrere Werkzeuge in Frage. Git\footnote{http://git-scm.com} ist ein verteiltes Sourcecode Management System und ist konzeptionell und hinsichtlich Benutzerfreundlichkeit besser als Subversion und CVS\footnote{Git kann offline verwendet werden, das Verschieben von Verzeichnissen führt nicht zu Problemen, etc.}. Es handelt sich um ein verteiltes Versionsverwaltungssysteme welches nicht grundsätzlich bedingt, dass es ein zentrales Repository gibt. Ein Repository befindet sich auf jedem Client. Trotzdem setzt man aber in der Regel auf ein Repository auf, in welches die Entwickler jederzeit commiten können. Eine empfehlenswerte Einführung in Git kann unter \cite{dilger201111}  gefunden werden.

Auf der Plattform Github\footnote{http://github.com} kann man Git-Projekte ``hosten''.  




\subsection{Continous Integration}
Continuous Integration Systeme dienen zur Steigerung der Softwarequalität. Sie machen dies in der Regel, indem sie alle Tests und den Gesamtbuild der Software periodisch - üblich ist jede Stunde - durchführen. Für die Analysesoftware gibt es bei der Evaluation einige Grundvoraussetzungen:
\begin{itemize}
	\item \textbf{Buildwerkzeug Maven:} Das System muss Maven als Build- und Automatisierungswerkzeug unterstützen.
	\item \textbf{Git Versionskontrolle:} Der Quelltext der Applikation muss via Git vom Sourcecode Repository ausgecheckt werden können.
	\item \textbf{Freie Lizenz:} Die Software muss mindestens frei verfügbar sein oder open-source.
\end{itemize}
In den letzten Jahren hat sich Hudson\footnote{Vor kurzem haben sich einige Entwickler von Hudson aufgrund von Streitigkeiten mit Oracle dazu entschieden, die Software weiter unter dem Namen Jenkins zu entwickeln.} in vielen Projekten durchgesetzt. Hudson ist eine open-source Continuous Integration Software die gegenüber anderen Systemen einige Vorteile mit sich bringt:
\begin{itemize}
\item Hudson ist open-source.
\item Hudson ist sehr einfach zu installieren und administrieren.
\item Hudson basiert auf einem Plugin-System und ist aufgrund dessen erweiterbar. Es gibt Plugins für die Integration von Maven-Projekten und den Zugriff auf Git Repositories.
\end{itemize}

\section{Konfigurationsmanagement}
Jedes Deployment entspricht einer Version im JIRA, unserem verwendeten Issue-Tracker (siehe Abschnitt \ref{issue_tracker}).
Ab der ersten Produktivschaltung entsprechen die Versionierung folgendem Muster:
\begin{center}
\fbox{\textbf{\textless Major\textgreater\textless Minor\textgreater .\textless Build\textgreater}}
\end{center}

\textbf{Ein Beispiel für eine Versionsnummer wäre 1.3.1. Gestartet wird bei der ersten produktiven Version mit 1.0.0}. 
\subsection{Pflege der einzelnen Versionen}
\begin{itemize}
\item \textbf{Build:} Bei jedem Bugfix- und / oder sonstigen ausserordentlichen Zwischenrelease wird die Version um eins hochgezählt.
\item \textbf{Minor:} Bei jedem Feature-Deployment während einem ordentlichen Release wird die Minor-Version um eins hochgezählt.
\item \textbf{Major:} Bei grossen Änderungen und / oder der Entwicklung von Zusatzmodulen wird die Major-Version um eins hochgezählt.

\end{itemize}


\subsection{Versionsverwaltung}
Das zentrale Repository dieses Projektes befindet sich auf Github\footnote{https://github.com/schmidic/bachelorthesis}. Es ist jederzeit verfügbar und kann von allen Clients (Entwickler, Continuous Integration) jederzeit angesprochen werden.
\subsubsection{Übersicht}
Das Repository besteht aus drei Branches:
\begin{itemize}
\item \textbf{master:} Auf dem Master-Branch befindet sich der aktuelle Entwicklungsstand. 
\item \textbf{release:} Nur vom Release-Branch wird in die Produktion deployt. Dadurch kann jederzeit am produktiven System ein Bugfix gemacht werden. Der Release-Branch entspricht der aktuell deployten produktiven Applikation.
\item \textbf{next:} Auf dem Next-Branch befinden werden zukünftige Features entwickelt, die noch nicht in die Produktion deployt werden sollen.
\end{itemize}

\subsubsection{Bugfix Entwicklung}
Ein Bugfix der Produktiven Applikation findet auf der Basis des \textit{release}-Branches statt. Dazu wird daraus allerdings ein neuer Branch erzeugt. Die Minor-Version für den Bugfix wird um eins inkrementiert. Ausgehend von einem Beispiel zweier Bugfixes für die Version 1.0.0 ist der Ablauf wie folgt:
\begin{enumerate}
\item Für den laufenden Release (Version 1.0.0) werden die Bugs (\#4321, \#4322) im Jira rapportiert.
\item Der Entwickler checkt den \textit{release}-Branche aus und erstellt daraus einen neuen Branch (\textit{bugfix-for-1.0.0}).
\item Der Entwickler inkrementiert die Maven-Version um eins (neu 1.0.1).
\item Der Entwickler erstellt eine neue Jira Version (1.0.1).
\item Der Entwickler repariert die Fehler, die Applikation wird getestet.
\item Die Bugs innerhalb des Jiras werden der neuen Version zugeordnet.
\item Der Entwickler mergt die Bugfixes zurück in den \textit{release}-Branch. Die Zuordnung der Commits mit den Jira-Bugs wird über den Kommentar mit dem Bug im Jira verlinkt. Die Anpassung des Kommentars der vorherigen Commits kann mittels einem Rebasing (siehe \cite{dilger201111}) gemacht werden.
\item Erstellen eines Tags für den Release
\item Release der neuen Version (1.0.1).
\end{enumerate}
Mit dem dedizierten \textit{release}-Branch kommt ein wesentlicher Vorteil zum Tragen: Man kann ein Bugfix Release in das Produktivsystem veranlassen der nur einen bestimmten Bugfix enthält, selbst wenn auf dem \textit{master} eventuell schon weitere Features entwickelt wurden.

In der Jira Versionsverwaltung wird für den Bugfix eine neue Version geführt. Sämtliche 

\subsubsection{Feature Entwicklung}
Die Entwicklung eines Features, welches in die Produktion einfliessen soll, wird auf dem Master-Branch durchgeführt. Um Konflikte mit dem \textit{release}-Branch zu vermeiden, wird nach jedem Release zumindest die Minor-Version des \textit{master}-Branches um eins erhöht.  Das Vorgehen ist wie folgt:
\begin{enumerate}
\item Für den laufenden Release gibt es die Feature-Requests (\#4351, \#4352) im Jira.
\item Der Entwickler checkt den \textit{master}-Branch aus und erstellt daraus einen neuen Branch (\textit{feature-1.1.0})
\item Der Entwickler inkrementiert die Maven-Version (1.1.0).
\item Der Entwickler erstellt eine neue Jira Version (1.1.0).
\item Der Entwickler erstellt das neue Feature.
\item Der Entwickler ordnet die Feature-Requests im Jira der neu erstellten Version zu.
\item Der Entwickler mergt den \textit{feature-1.1.0}-Branch zurück in den \textit{master}-Branch.  Die Zuordnung der Commits mit den Jira-Bugs wird über den Kommentar mit dem Bug im Jira verlinkt. Die Anpassung des Kommentars der vorherigen Commits kann mittels einem Rebasing (siehe \cite{dilger201111}) gemacht werden.
\end{enumerate}