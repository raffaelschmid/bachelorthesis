\chapter{Konzept (Infrastruktur)}\label{konzept_3}
\section{Versionskontrolle}
Zur Versionsverwaltungssysteme kommen mehrere Werkzeuge in Frage. Git\footnote{http://git-scm.com} ist ein verteiltes Sourcecode Management System und ist konzeptionell und hinsichtlich Benutzerfreundlichkeit besser als Subversion und CVS\footnote{Git kann offline verwendet werden, das Verschieben von Verzeichnissen führt nicht zu Problemen, etc.}. Auf der Plattform Github\footnote{http://github.com} kann man öffentliche Projekte gratis ``hosten''.

\section{Build-Automatisierung}
Die Automatisierung des Software-Builds ist hinsichtlich der Integration in ein Continuous Integration System wichtig. Zusätzlich entfallen so zeitaufwändige Tasks wie die Paketierung und das Deployment der Applikation.
Als Werkzeug zum automatisierten Build der Software wird Maven Tycho\footnote{Im Bereich der Eclipse Rich Client Entwicklung kann entweder PDE Build, ein auf Apache Ant basiertes Build-System für Eclipse RCP Applikationen\cite{vogelZapfPdeBuild} oder die Maven-Integration Tycho (http://tycho.sonatype.org) verwendet werden.} verwendet.

Tycho ist relativ neu und bringt im Vergleich mit dem PDE Build einige Vorteile mit sich:
\begin{itemize}
	\item Maven folgt dem Prinzip ``Convention over Configuration''\footnote{Das Prinzip ``Convention over Configuration'' hat zur Folge, dass im Wesentlichen nur von den Standardeinstellungen abweichende Werte konfiguriert werden müssen.} - die Konfiguration des Builds wird dadurch wesentlich einfacher.
	\item Maven ist de facto Standard bei den Build-Werkzeugen.
\end{itemize}

\section{Continous Integration}
Continuous Integration Systeme dienen zur Steigerung der Softwarequalität. Sie machen dies in der Regel, indem sie alle Tests und den Gesamtbuild der Software periodisch - üblich ist jede Stunde - durchführen. Für die Analysesoftware gibt es bei der Evaluation einige Grundvoraussetzungen:
\begin{itemize}
	\item \textbf{Buildwerkzeug Maven:} Das System muss Maven als Build- und Automatisierungswerkzeug unterstützen.
	\item \textbf{Git Versionskontrolle:} Der Quelltext der Applikation muss via Git vom Sourcecode Repository ausgecheckt werden können.
	\item \textbf{Freie Lizenz:} Die Software muss mindestens frei verfügbar sein oder open-source.
\end{itemize}
In den letzten Jahren hat sich Hudson\footnote{Vor kurzem haben sich einige Entwickler von Hudson aufgrund von Streitigkeiten mit Oracle dazu entschieden, die Software weiter unter dem Namen Jenkins zu entwickeln.} in vielen Projekten durchgesetzt. Hudson ist eine open-source Continuous Integration Software die gegenüber anderen Systemen einige Vorteile mit sich bringt:
\begin{itemize}
\item Hudson ist open-source.
\item Hudson ist sehr einfach zu installieren und administrieren.
\item Hudson basiert auf einem Plugin-System und ist aufgrund dessen erweiterbar. Es gibt Plugins für die Integration von Maven-Projekten und den Zugriff auf Git Repositories.
\end{itemize}



\section{Issue Tracker}
Als Issue Tracker wird Jira verwendet. Es handelt sich dabei um eine kostenpflichtige aber relativ günstige Software für das Issue-Tracking.


