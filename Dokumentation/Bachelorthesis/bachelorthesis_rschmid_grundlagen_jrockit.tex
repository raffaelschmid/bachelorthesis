\chapter{Garbage Collection JRockit}\label{jrockit garbage collection}
Obwohl sich die Garbage Collection Algorithmen und Strategien der JRockit VM gegenüber der HotSpot VM in der Tiefe stark unterscheiden, haben sie Oberflächlich betrachtet viele Gemeinsamkeiten. Dieser Abschnitt geht auf die Eigenheiten der JRockit Virtual Machine ein. 

\section{Algorithmen}
Die Grundlage der JRockit Garbage Collection bildet der Tri-Coloring Mark \& Sweep Algorithmus (siehe Abschnitt \titleref{tri-coloring mark and sweep}). Er wurde hinsichtlich besserer Parallelisierbarkeit und der optimalen Verwendung der Anzahl Threads optimiert. Die Garbage Collection der JRockit VM kann mit oder ohne Generationen arbeiten - so gibt es die beiden Algorithmen ``Concurrent Mark \& Sweep'' und ``Parallel Mark \& Sweep'' in beiden Ausführungen Generational und Single:

\begin{itemize}
	\item Generational Concurrent Mark \& Sweep
	\item Single Concurrent Mark \& Sweep
	\item Generational Parallel Mark \& Sweep	
	\item Single Parallel Mark \& Sweep
\end{itemize}

\subsection{Optimierungen}
\subsubsection{Verwendung von threadlokalem Speicher}
Im Unterschied zum normalen Tri-Coloring Algorithmus verwendet der Algorithmus der JRockit, egal ob parallel oder concurrent, zwei Sets für die Markierung der Objekte. In einem werden die grauen und schwarzen Objekte gespeichert, im anderen die weissen. Die Trennung zwischen grau und schwarz wird gemacht, indem die grauen Objekte in threadlokalen Queues jedes Garbage Collection Threads gespeichert werden. 

Die Verwendung von threadlokalem Speicher hat hinsichtlich den folgenden Punkten einen Vorteil:\cite[S. 79]{lagergren2010oracle}:
\begin{itemize}
	\item Threadlokaler Speicher führt zu einer besseren Parallelisierbarkeit, da Zugriffe darauf nicht synchronisiert werden müssen.
	\item Threadlokaler Speicher kann voraus geladen (Prefetching) werden, was die Geschwindigkeit des Algorithmus als Ganzes erhöht.
\end{itemize}

Bei der Verwendung der Concurrent Algorithmen wird parallel dazu auch Speicher für neue Objekte alloziert. Diese neuen Objekte werden in einem sogenannten Live-Set getrackt, damit sie in der Sweep Phase - obwohl nicht markiert - nicht gelöscht werden.

\subsubsection{Kompaktierung}
Aufgrund der im Speicher stattfindenden Fragmentierung wird bei jeder Old Collection eine Kompaktierung des Speichers durchgeführt. Während der Sweep Phase werden die Objekte auf dem Heap umher kopiert, so dass danach wieder grössere freie Speicherbereiche existieren.

\subsection{Übersicht und Auswahl Algorithmen}\label{auswahl_algorithmen}
Algorithmen können direkt selektiert werden. Manchmal passiert es aber trotzdem, dass der Garbage Collection Algorithus im laufenden Betrieb geändert wird.
  \begin{longtable}{|p{2cm}|p{2.8cm}|l|c|c|c|c|c|}
    \hline
  \textbf{Alias} & \textbf{Aktivierung}& \textbf{Generation} & \textbf{Pause} &\textbf{Durchs.} & \textbf{Heap} & \textbf{Mark} & \textbf{Sweep} \\\hline
  singlecon, singleconcon & -Xgc:singlecon\newline-Xgc:singleconcon& - &++&--& single & konk. & konk\\\hline

  gencon,\newline genconcon& -Xgc:gencon\newline-Xgc:genconcon &Old, Young&++&-- & gen & konk. & konk.\\\hline	
 
  singlepar, \newline singleparpar& -Xgc:singlepar\newline-Xgc:singleparpar & - & -- & ++ & single & parallel & parallel \\\hline
  genpar, \newline genparpar& -Xgc:genpar\newline-Xgc:genparpar & Old, Young & -- & ++ & gen& parallel & parallel \\\hline
  genconpar &-Xgc:genconpar&Old, Young&+&+&gen& konk. & parallel \\\hline
  genparcon &-Xgc:genparcon&Old, Young&+&+&gen& parallel & konk. \\\hline
  singleconpar &-Xgc:singleconpar&-&+&+&single&konk. & parallel \\\hline
  singleparcon& -Xgc:singleparcon&-&+&+&single&parallel & konk. \\\hline
\caption{Auswahl des Garbage Collection Algorithmus}
  \end{longtable}

\section{Nebenläufige Algorithmen}
Bei den Concurrent Mark \& Sweep Algorithmen handelt es sich eigentlich um ''mostly concurrent`` Algorithmen. Das heisst, sie finden nicht in allen Phasen konkurrierend zur Applikation statt. Damit ein Teil der \textbf{Markierungsphase} konkurrierend statt finden kann, ist sie in vier Phasen aufgeteilt:
\begin{itemize}
	\item Initial Marking (Nicht konkurrierend): Hier wird das Root Set zusammengestellt.
	\item Concurrent Marking (konkurrierend): Mehrere Threads gehen nun diesen Referenzen nach und markieren die Objekte als lebendig.
	\item Preclean (konkurrierend): Änderungen im Heap während den vorherigen Schritten werden nachgeführt, markiert.
	\item Final Marking (Nicht konkurrierend): Änderungen im Heap während der Precleaning Phase werden nachgeführt, markiert.
\end{itemize}

Die \textbf{Sweep Phase} findet ebenfalls konkurrierend zur Applikation statt. Im Gegensatz zur HotSpot VM ist sie aber in zwei Schritte aufgeteilt. Als erstes wird die erste Hälfte des Heaps von toten Objekten befreit, während dieser Phase können Threads Speicher in der zweiten Hälfte des Heaps allozieren. Nach einer kurzen Synchronisationspause findet das Sweeping auf dem zweiten Teil des Heaps statt.

\section{Parallele Algorithmen}
Bei den parallelen Mark \& Sweep Algorithmen findet die Garbage Collection mit allen verfügbaren Prozessoren statt. Dazu werden aber alle Threads der Applikation gestoppt.


\section{Garbage Collection Modi}\label{gc_modus}
Anstelle einer direkten Auswahl eines Algorithmus (siehe Abschnitt \ref{auswahl_algorithmen}), kann auch eine grundlegende Strategie angegeben werden. Die Virtual Machine entscheided dann heuristisch, welcher spezifische Algorithmus gewählt wird - in der Regel entscheided er insbesondere, ob er den Heap in Generationen aufteilen soll. Die verfügbaren Modi entsprechen den Tuningzielen die es für die Garbage Collection gibt (siehe Abschnitt \ref{garbage_collection_tuning}):
\begin{itemize}
	\item \textbf{Durchsatz (throughput):} Optimiert die Garbage Collection hinsichtlich möglichst grossem Durchsatz. Um dieses Ziel zu erreichen, werden parallele Algorithmen verwendet. Sie laufen nicht-konkurrierend mit der Applikation und führen zu kurzen Pausenzeiten (Stop-the-World Pausen). In der restlichen Zeit stehen der Applikation allerdings sehr viel Ressourcen zur Verfügung.
	\item \textbf{Pausenzeit (pause time):} Die Optimierung der Pausenzeiten hat zur Folge, dass die Applikation durch möglichst wenig Stop-the-World Pausen angehalten wird. Das ist insbesondere für Client-Server Applikationen wichtig. Das Ziel wird mit der Verwendung eines Concurrent Garbage Collection Algorithmus erreicht.
	\item \textbf{Determinismus (deterministic):} Optimiert den Garbage Collector auf sehr kurze und deterministische Garbage Collection Pausen.  
\end{itemize}
  \begin{longtable}{|p{2.5cm}|p{3.1cm}|p{4.5cm}|p{4cm}|}\hline
  \textbf{Alias} & \textbf{Aktivierung} & \textbf{Beschreibung} & Einstellungsmöglichkeiten\\\hline
  troughput & -Xgc:throughput & Der Garbage Collector wird auf maximalen Durchsatz der Applikation eingestellt. Er arbeitet so effektiv wie möglich und erhält entsprechend viele Java-Threads, was zu kurzen Pausen der Applikation führen kann. & \\\hline
  pausetime & -Xgc:pausetime & Der Garbage Collector wird auf möglichst kurze Pausen eingestellt. Das bedeutet, dass ein konkurrierender Algorithmus verwendet wird, der insgesamt etwas mehr CPU Ressourcen benötigt. & -XpauseTarget=value (default 200msec)\\\hline
  deterministic & -Xgc:deterministic & Der Garbage Collector wird auf eine möglichst kurze und deterministische Pausenzeit eingestellt. &-XpauseTarget=value (default 30msec)\\\hline
    \caption{Übersicht der Garbage Collection Modi}
\end{longtable}

\section{Garbage Collection Logdateien}\label{jrockitgclog}
Die Auswertung der Garbage Collection kann auf Basis von Logdateien gemacht werden. Das Format dieser Logdateien hängt mitunter auch von den aktivierten Log-Modulen ab. Der Aufbau der Logdateien und die in der Analyse verwendeten Daten werden in diesem Abschnitt genauer beleuchtet.

\subsection{Aktivierung Log Ausgaben}
Standardmässig macht die JRockit keine Angaben darüber, wie sie die Garbage Collection durchführt. Es besteht aber durch die Aktivierung des entsprechenden Log-Modules die Möglichkeit, an diese Informationen zu gelangen. Die Ausgaben werden dann auf die Standard Ausgabe oder optional (Argument -Xverboselog:logdatei.log) in eine Datei geschrieben. Das Format für die Kommandozeile ist wie folgt:

\begin{lstlisting}[caption=Format Aktivierung Log Modul]
-Xverbose:<modul>[=log_level]
\end{lstlisting}

Um das Memory Log Modul (gibt Informationen über die Garbage Collection aus) zu aktivieren muss man folgendes Argument übergeben:
\begin{lstlisting}[caption=Garbage Collection Log (Info)]
-Xverbose:memory
\end{lstlisting}

Die Umleitung der Ausgaben in eine separate Logdatei kann folgendermassen angegeben werden:
\begin{lstlisting}[caption=Garbage Collection Log (Info) - Umleitung in gc.log]
-Xverbose:memory -Xverboselog:gc.log 
\end{lstlisting}


Zusätzlich kann pro Log-Modul der Log-Level angepasst werden:
\begin{lstlisting}[caption=Einstellung des Log-Levels]
-Xverbose:memory=debug -Xverboselog:gc.log 
\end{lstlisting}
Als Log-Levels stehen folgende Werte zur Verfügung: quiet, error, warn, info, debug, trace. Für weiter Informationen siehe \cite{oracleJRockitR28CLR}.

\subsection{Beispiel einer Garbage Collection Logdatei}
Die ersten Zeilen einer Log-Datei sind auf der folgenden Seite ersichtlich:
\begin{landscape}\label{logexample}
\begin{lstlisting}[caption=Garbage Collection Logdatei]
[INFO ][memory ] GC mode: Garbage collection optimized for throughput, strategy: Generational Parallel Mark & Sweep.
[INFO ][memory ] Heap size: 65536KB, maximal heap size: 1048576KB, nursery size: 32768KB.
[INFO ][memory ] <start>-<end>: <type> <before>KB-><after>KB (<heap>KB), <time> ms, sum of pauses <pause> ms.
[INFO ][memory ] <start>  - start time of collection (seconds since jvm start).
[INFO ][memory ] <type>   - OC (old collection) or YC (young collection).
[INFO ][memory ] <end>    - end time of collection (seconds since jvm start).
[INFO ][memory ] <before> - memory used by objects before collection (KB).
[INFO ][memory ] <after>  - memory used by objects after collection (KB).
[INFO ][memory ] <heap>   - size of heap after collection (KB).
[INFO ][memory ] <time>   - total time of collection (milliseconds).
[INFO ][memory ] <pause>  - total sum of pauses during collection (milliseconds).
[INFO ][memory ]            Run with -Xverbose:gcpause to see individual phases.
[INFO ][memory ] [OC#1] 0.843-0.845: OC 428KB->78423KB (117108KB), 0.003 s, sum of pauses 1.614 ms, longest pause 1.614 ms.
[INFO ][memory ] [OC#2] 1.393-1.442: OC 78449KB->156488KB (233624KB), 0.049 s, sum of pauses 47.104 ms, longest pause 47.104 ms.
[INFO ][memory ] [YC#1] 1.494-1.496: YC 156524KB->156628KB (233624KB), 0.002 s, sum of pauses 1.670 ms, longest pause 1.670 ms.
[INFO ][memory ] [YC#2] 1.496-1.496: YC 156652KB->156755KB (233624KB), 0.001 s, sum of pauses 0.605 ms, longest pause 0.605 ms.
[INFO ][memory ] [YC#3] 1.497-1.497: YC 156780KB->156884KB (233624KB), 0.001 s, sum of pauses 0.602 ms, longest pause 0.602 ms.
[INFO ][memory ] [YC#4] 1.497-1.498: YC 156908KB->157011KB (233624KB), 0.001 s, sum of pauses 0.592 ms, longest pause 0.592 ms.
\end{lstlisting}
\end{landscape}

\subsection{Log Module}\label{logmodule}

Die folgende Tabelle beschreibt die für die Auswertung der Garbage Collection relevanten Module (alphabetisch sortiert). 

\begin{longtable}{|p{4cm}|p{9cm}|p{2cm}|}
  \hline
  \textbf{Modul} & \textbf{Beschreibung} & \textbf{Relevanz}\\\hline
  alloc & Informationen betreffend Speicher Allokation und ``Out-of-Memory'' & niedrig \\\hline
  compaction & zeigt abhängig vom Garbage Collection Algorithmus Informationen zur Kompaktierung& niedrig \\\hline
  gcheuristic & zeigt Informationen der Garbage Collection Heuristik & mittel \\\hline
  gcpause & zeigt wann welche Pausen zwecks Garbage Collection gemacht wurden, wie lange sie gedauert haben. & mittel \\\hline
  gcreport & zeigt verschiedene Auswertungen (Anzahl Collections, Anzahl promotete Objekte, maximal promotete Objekte per Zyklus, totale Zeit, Durchschnittszeit, Maximale Zeit) der Garbage Collection zum aktuellen Lauf& niedrig \\\hline
  memory & zeigt Informationen zum Memory Management System wie Start-, Endzeitpunkt der Collection, Speicher bevor, nach Collection, Grösse des Heaps, etc.& \textbf{sehr hoch} \\\hline
  memdbg & zeigt debug Informationen zu Speicher belange& \textbf{hoch} \\\hline
  systemgc & zeigt Garbage Collections die durch System.gc() gestartet wurden. & niedrig \\\hline
    \caption{Beschreibung der verschiedenen relevanten Log Modulen}
\end{longtable}