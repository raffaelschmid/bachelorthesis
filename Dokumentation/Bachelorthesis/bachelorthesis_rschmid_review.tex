\chapter{Review und Ausblick}\label{review}
Die Implementation beinhaltet alle funktionalen Anforderungen. Die Qualitäts-anforderungen \textit{Testabdeckung} (QRQ-S-02) und \textit{Internationalisierung} (QRQ-S-03) sind teilweise umgesetzt. Dieser Abschnitt beleuchtet den Scope der Software und zeigt auf, wo weiteres Entwicklungspotential liegt.

\section{Was das Tool leistet}
Wenn die Auslastung einer CPU konstant hoch ist aber nicht durch das System verursacht wird, kann unter Umständen das Tuning des Garbage Collectors Sinn machen. 
Zur Auswertung einer laufenden virtuellen Maschine, kann auch die Software JRockit Mission Control von Oracle verwendet werden. Oft ist allerdings der Zugriff via Mission Control auf den Server nicht möglich\footnote{Es gibt Firewall-Regeln oder man befindet sich ausserhalb des Server-Subnetzes}. In diesem Fall müssen zur Analyse der Garbage Collection die erstellten Logdateien verwendet werden. Mit der Analysesoftware kann diese Aufgabe vereinfacht werden, indem beispielsweise der Verlauf der Garbage Collection Zeiten oder des durch die Applikation benötigten Arbeitsspeichers angeschaut werden kann. Wichtig dabei können auch Metriken wie Durchsatz oder die durchschnittliche Pausenzeit sein.

Aktuell werden die Log-Einträge des Memory-Moduls  (Log-Level: Info) ausgewertet. Das sind die wichtigsten Ausgaben, die im Zusammenhang mit der Garbage Collection geschrieben werden. Um eine Auswertung einer Logdatei zu machen, muss eine Logdatei mit den Ausgaben der Speicherverwaltung erstellt werden\footnote{Die Aktivierung des Loggers kann über das Argument -Xverbose:memory gemacht werden}.

Version R28 ist die neuste Version der JRockit. Die Analysesoftware wurde auf das Format dieser Virutellen Machine ausgerichtet. Die Analyse von Logdateien älterer Versionen ist momentan noch nicht möglich.

\section{Ausblick}
\subsection{Funktionsumfang}
Beim Performance Tuning im Bereich des Garbage Collectors vergleicht man oft die Implikation einer Anpassung der Konfiguration. Aktuell können zwar verschiedene Logdateien gleichzeitig geöffnet werden, der Vergleich von Kurven im gleichen Diagramm ist noch nicht möglich.

Weiter ist man beim Performance Tuning in der Regel nicht an den Informationen über die ganze Zeit interessiert, die Möglichkeit zur Wahl von Start- und Endzeitpunkt der auszuwertenden Daten wäre deshalb wünschenswert.

\subsection{Weitere Daten}
Aktuell werden nur die Log-Einträge des Memory-Moduls im Log-Level INFO berücksichtigt. Wie der nachfolgende Auszug einer Logdatei zeigt, würden in den Debug-Einträgen wichtige Informationen stehen.

\begin{lstlisting}[caption=Garbage Collection Log (Debug Informationen)]
[INFO ][alloc  ] [OC#1] Satisfied 0 object and 0 tla allocations. Pending requests went from 1 to 1.
[DEBUG][memory ] [OC#1] Initial marking phase promoted 3620 objects (206KB).
[DEBUG][memory ] [OC#1] Starting concurrent marking phase (OC2).
[DEBUG][memory ] [OC#1] Concurrent mark phase lasted 0.235 ms.
[DEBUG][memory ] [OC#1] Starting precleaning phase (OC3).
[DEBUG][memory ] [OC#1] Precleaning phase lasted 0.249 ms.
[DEBUG][memory ] [OC#1] Starting final marking phase (OC4).
[INFO ][nursery] [OC#1] Young collection started. This YC is a part of OC#1 final marking.
\end{lstlisting}

Einige Beispiele für solche Informationen sind:
\begin{itemize}
	\item \textbf{Gründe, warum eine Garbage Collection gestartet wurde:  }Die erste Zeile im Listing oben zeigt, dass es hängige Anfragen für die Allokation von Speicher gibt, was zu einer Garbage Coolection führt.
	\item \textbf{Dauer der einzelnen Garbage Collection Phasen (Initial Marking, Precleaning, Final Marking): } Nicht alle diese Phasen laufen beispielsweise konkurrierend ab. Im Sinne von möglichst kurzen Pausenzeiten ist es deshalb interessant, wie lange die einzelnen Phasen und im speziellen die Final Marking Phase gedauert haben. 
\end{itemize}

\subsection{Andere Log-Formate}
Wie bereits erwähnt, können die Logs der Version R27 noch nicht ausgewertet werden. Weil diese Version noch an vielen Orten im Einsatz ist, muss bald die Kompatibilität mit Release 27 hergestellt werden.

Im Bereich der HotSpot virtual Machine ist seit Version 1.6.0\_14 des Java Runtime Environments eine Vorversion des G1 Garbage Collectors\footnote{G1 ist auch unter dem Namen Garbage First Garbage Collector bekannt.} verfügbar. Die Funktionsweise dieses Algorithmus unterscheidet sich stark von den bisherigen Versionen des Mark \& Sweep Algorithmus. Die Logdateien dieses Collectors können aktuell noch durch keine Software ausgewertet werden\cite{langerkreftJavaCore}. Mit der Auswertungsmöglichkeit für diesen Algorithmus könnte man sich auf dem Markt besser positionieren.






