\begin{landscape}
\section{Funktionale Anforderungen}
Aus den im Abschnitt \titleref{use_cases} definierten Anforderungen ergibt sich die abschliessende Liste der funktionalen Anforderungen:
\begin{longtable}{|p{1.8cm}|p{0.7cm}|p{2.5cm}|p{6cm}|p{1.6cm}|p{3cm}|p{0.9cm}|}
    \caption{Funktionale Anforderungen}\\\hline
   \textbf{Identifik.} & \textbf{Vers.}& \textbf{Titel} & \textbf{Beschreibung} & \textbf{Use Case} & \textbf{Abnahmekriter.} &\textbf{Prio.}\\\hline
   FRQ-A01 & 1.0 & Installation & Software muss als Erweiterung in der Entwicklungsumgebung\footnote{Die Wahl der Entwicklungsumgebung respektive des Frameworks befindet sich im Abschnitt \titleref{selection_rcp_fw}.} installiert werden können.  & UC-A01 & Entwickler mit durchschnittlichen Kenntnissen benötigt dafür weniger Zeit als 5 Minuten. & hoch  \\\hline

   FRQ-A02 & 1.0 & Updaten & Die Software kann mit geringem Aufwand aktualisiert werden. & UC-A02 & Entwickler mit durchschnittlichen Kenntnissen benötigt dafür weniger als Zeit als 2 Minuten. & mittel  \\\hline

  FRQ-A03 & 1.0 & Garbage Collection Log Datei importieren & Garbage Collection Log Datein der JRockit Virtual Machine R28 können importiert werden und stehen zur Analyse bereit. & UC-A03 & Der Importprozess bei einer Datei mit 100000 Zeilen dauert weniger als 10 Sekunden. & hoch  \\\hline

  FRQ-A04 & 1.0 & Garbage Collection Log Datei analysieren & Analysiert wird auf Basis der importierten Daten. Die Software ermöglicht das Öffnen im Standard-Report (Standard-Profil) oder benutzerdefinierten (Benutzerdefiniertes Profil)& UC-A04 & Das Öffnen des Analysefensters für eine Datei mit 100000 Zeilen dauert nach dem Einleseprozess weniger als 5 Sekunden. & hoch  \\\hline

   FRQ-A04.1 & 1.0 & Übersicht Garbage Collection & Der erste Tab des Analysefensters zeigt die aggregierten Daten der Garbage Collection. & UC-A04.1 & - & hoch \\\hline

   FRQ-A04.2 & 1.0 & Heap Benutzung & Der zweite Tab des Analysefensters zeigt den Speicherbedarf über die Zeit. & UC-A04.2 & - & hoch 
 \\\hline

   FRQ-A04.3 & 1.0 & Dauer Garbage Collection & Der dritte Tab des Analysefensters zeigt die Dauer der einzelnen Garbage Collections über die Zeit. & UC-A04.3 & - & mittel 
 \\\hline

   FRQ-A05 & 1.0 & Benutzerdefinierte Auswertung (Profil) & Benutzer kann eigenes Analysefenster (Profil) mit unterschiedlichen Charts definieren. & UC-A05 & - & niedrig \\\hline

   FRQ-A05.1 & 1.0 & Profil speichern & Definiertes Profil wird automatisch gespeichert und bleibt über die Dauer der Sitzung bestehen. & UC-A05.1 & - & niedrig \\\hline

  FRQ-A05.1 & 1.0 & Profil exportieren & Profil kann in Textdatei exportiert werden. & UC-A05.1 & - & niedrig \\\hline

  FRQ-A05.1 & 1.0 & Profil importieren & Profil kann aus Textdatei importiert werden. & UC-A05.1 & - & niedrig \\\hline

  FRQ-A06 & 1.0 & Hilfesystem &  Dem Benutzer werden eine indexbasierte\footnote{Generelle Hilfe mit Informationen zur Garbage Collection,Vorgehensweise bei Performance Problemen, alternative Werkzeuge, etc.} und eine kontextsensitive\footnote{Hilfe zur aktuellen View oder Aktion des Benutzers} Hilfe zur Verfügung gestellt. & UC-A06 & - & niedrig \\\hline

  FRQ-E01 & 1.0 & Neuer Software-Release erstellen & Der Benutzer kann die Software mit Patches oder zusätzlichen Features jederzeit neu deployen. Der Anwender erhält die neue Software via den Update-Mechanismus.& UC-E01 & - & niedrig \\\hline

\end{longtable}
\end{landscape}

\begin{landscape}
\section{Qualitätsanforderungen}
\subsection{Software}
\begin{longtable}{|p{1.8cm}|p{0.7cm}|p{2.5cm}|p{8cm}|p{3cm}|p{0.9cm}|}
    \caption{Qualitätsanforderungen Software}\\\hline
   \textbf{Identifik.} & \textbf{Vers.}& \textbf{Titel} & \textbf{Beschreibung} & \textbf{Abnahmekriter.} &\textbf{Prio.}\\\hline
   QRQ-S-01 & 1.0 & Erweiterbarkeit & Nebst der Analyse von Garbage Collection Logs der JRockit Virtual Machine sollen später auch andere Formate unterstützt sein. & Erweiterung um ein weiteres Log-Format soll den Aufwand von 5 PT\footnote{Personentage} nicht überschreiten. & mittel \\\hline
   QRQ-S-02 & 1.0 & Testabdeckung & Um den langfristigen Erfolg dieser Software zu gewährleisten muss eine entsprechende Testabdeckung vorhanden sein - dies um insbesondere die Regression zu vermeiden. & Angestrebte Test-Coverage: 80\% & gering \\\hline
  QRQ-S-03 & 1.0 & Internationali-sierung & Die Sprachelemente der Software (Labels, Titel, Texte) werden als Ressourcen definiert, was die spätere Erweiterung ermöglicht. & - & gering\\\hline

\end{longtable}
\subsection{Basisframework}\label{anforderungen_framework}
\begin{longtable}{|p{1.8cm}|p{0.7cm}|p{2.5cm}|p{8cm}|p{3cm}|p{0.9cm}|}
    \caption{Qualitätsanforderungen Basisframework}\\\hline
   \textbf{Identifik.} & \textbf{Vers.}& \textbf{Titel} & \textbf{Beschreibung} & \textbf{Abnahmekriter.} &\textbf{Prio.}\\\hline
   QRQ-F-01 & 1.0 & Verbreitung & Die Software wird als Erweiterung für eine Entwicklungsumgebung bereitgestellt. Auch weil die Entwickler-Community auf diesen Plattformen am grössten ist, spielt die Verbreitung des verwendeten Frameworks eine grosse Rolle.   & - & hoch \\\hline

   QRQ-F-02 & 1.0 & Plattform-unabhängig &  Die Software soll auf den gängigsten Betriebssystemen Windows und Apple OSX laufen. & - & mittel \\\hline

   QRQ-F-03 & 1.0 & Lokalisation & Framework muss Unterstützung für Lokalisation bereitstellen. & - &gering \\\hline


\end{longtable}
\end{landscape}


