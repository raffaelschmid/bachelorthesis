\section{Teilnehmer}
\begin{itemize}
	\item Olaf Stern (Schulleiter)
	\item Matthias Bachmann (Betreuer)
	\item Raffael Schmid (Student)
\end{itemize}
\section{Einleitung}
Mit der Bachelorthesis möchte der Student Raffael Schmid sein Studium an der Hochschule für Technik abschliessen. Als Thema hat er die Konzeption und Implementation eines Auswertungswerkzeugs für Garbage Collection Logs der JRockit Virtual Machine gewählt. 

\section{Ablauf}
\begin{enumerate} 
\item \textbf{Begrüssung} - Die Teilnehmer begrüssen sich und definieren kurz den Ablauf des Kickoffs.
\item \textbf{Präsentation Thema} - Der Student Raffael Schmid stellt Thema, Ausgangslage, Ziele der Arbeit, Aufgabenstellung und erwartete Resultate basierend auf der eingegebenen Aufgabenstellung vor.
\item \textbf{Überprüfung "Checkliste für die Aufgabenstellung einer Bachelorarbeit"} - Anhand der Checkliste wird festgestellt, dass die Aufgabenstellung alle deren Punkte erfüllt.
\item \textbf{Diskussionsrunde} - Die Teilnehmer diskutieren verschiedene Punkt. Siehe dazu \titleref{besonderes}.
\end{enumerate}

\section{Besonderes}\label{besonderes}
\subsection{Schwerpunkt konzeptioneller Teil}
Seitens Schulleitung wurde nochmals erwähnt, dass der Schwerpunkt der Bachelorarbeit auf dem konzeptionellen Teil liegt.
\subsection{Wissenschaftliche Arbeitstechnik}
Bei Problemen respektive Entscheiden während der Umsetzung sollen die verschiedenen Möglichkeiten aufgezeigt werden. Vor- und Nachteile der verschiedenen Lösungsansätze sollen dokumentiert und die resultierenden Entscheidungen nachvollziehbar sein.
\subsection{Zusammenarbeit mit dem Betreuer}
Die Kommunikation von Betreuer und Student findet massgeblich an folgenden Terminen statt:
\begin{enumerate}
	\item Mitte bis Ende August: Informelles Meeting Überprüfung Requirements Engineering
	\item Mitte bis Ende September: Design Review (offizieller Termin)
	\item Mitte Dezember: Abgabe, Präsentation (offizieller Termin)
\end{enumerate}
\subsection{Abgabetermin}
Im neuen Reglement ist der Abgabetermin der Bachelorthesis mit 6 Monaten nach Freigabe der Aufgabenstellung definiert. Da es vorher in den Reglementen kleine inkonsistenten gab (als Starttermin wurde auch Kickoff-Datum erwähnt), wurde der Start-Termin auf den 22. Juni 2011 (Tag des Kickoffs) festgelegt.
